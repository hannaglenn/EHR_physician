\documentclass[11pt]{article}
\usepackage[utf8]{inputenc}
\usepackage{amsmath,amssymb,hyperref,array,xcolor,multicol,verbatim,mathpazo}
\usepackage[normalem]{ulem}
\usepackage[pdftex]{graphicx}
\usepackage{fullpage}
\usepackage{import}
\usepackage{adjustbox}
\usepackage{booktabs}
\usepackage[font=footnotesize,labelfont=bf]{caption}
\captionsetup{justification=raggedright,singlelinecheck=false}


\usepackage[backend=biber,style=authoryear,
sorting=ynt,citestyle=authoryear]{biblatex}
\addbibresource{papercitations.bib}
\usepackage{setspace}
\onehalfspacing
\addtolength{\skip\footins}{2pc plus 5pt}

\title{Electronic Health Records and Patient Access to Care: Evidence from Physician Labor Market Decisions}
\author{Hanna Glenn}
%\date{\today}

\DeclareLabeldate[online]{%
  \field{date}
  \field{year}
  \field{eventdate}
  \field{origdate}
  \field{urldate}
}

\begin{document}

\maketitle

Computerized medical records and the systems that encompass them came with expectations of revolutionizing the U.S. healthcare system. Former President Obama stated in 2009, “To improve the quality of our health care while lowering its cost, we will make the immediate investments necessary to ensure that, within five years, all of America’s medical records are computerized.” (\cite{presquote}). The movement towards digitization in health care was expected to have immediate and substantial impacts on the both quality and cost of care. A 2005 study estimated hundreds of billions of dollars saved if health information technology were to be fully implemented (\cite{hillestad2005}). Such expectations led the U.S. government to incentivize the use of electronic health care records (EHRs) in hospitals with the the Health Information Technology for Economic and Clinical Health (HITECH) Act in 2008 (\cite{hitech}). The movement towards EHR use is evidenced in hospitals from 2008 onward; the percentage of hospitals with the capability of using a basic EHR system went from 9 percent in 2008 to 84 percent in 2015 (\cite{stats}).

A large area of research has been devoted to whether EHRs accomplished this simultaneous cost reduction and quality improvement. Economic research has concluded that neither were realized to the extent that was expected. The median patient received the same quality of care and some hospitals were still seeing cost increases from EHRs even six years after implementation. This raises the question- did any benefit come from the push towards EHR implementation? Why were the main goals not realized? An important facet to the answer may be an aspect not yet considered: patient access to care. This paper seeks to understand the effect of EHR implementation on patient access to care through the channel of physician labor market decisions. 






\end{document}