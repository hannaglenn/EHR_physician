\documentclass[12pt]{article}
\usepackage[utf8]{inputenc}
\usepackage{amsmath,amssymb,hyperref,array,xcolor,multicol,verbatim,mathpazo}
\usepackage[normalem]{ulem}
\usepackage[pdftex]{graphicx}
\usepackage{fullpage}
\usepackage{import}
\usepackage{adjustbox}
\usepackage{booktabs}
\usepackage{bbm}
\usepackage[font=normalsize,labelfont=bf]{caption}
%\captionsetup{justification=raggedright,singlelinecheck=false}
\usepackage{subcaption}

\usepackage[backend=biber,style=authoryear,
sorting=ynt,citestyle=authoryear]{biblatex}
\addbibresource{papercitations.bib}
\usepackage{setspace}
\onehalfspacing
\addtolength{\skip\footins}{2pc plus 5pt}

\usepackage{geometry}
 \geometry{,
 left=25.4mm,
 top=25.4mm,
 right=25.4mm,
 bottom=25.4mm
 }
 

\title{Labor Markets and Technological Change: Evidence from Electronic Health Records}
\author{Hanna Glenn}
%\date{\today}

\DeclareLabeldate[online]{%
  \field{date}
  \field{year}
  \field{eventdate}
  \field{origdate}
  \field{urldate}
}
\usepackage{times}
\begin{document}

\maketitle

\begin{abstract}
    Technological innovation has affected workers and labor markets for decades, yet the effects of modern technologies in some vocations with unique incentives are still not understood. I add to the understanding of technology implementation's affect on highly technical jobs by investigating changes in physician behavior as a result of a major technology shift in U.S. health care, electronic health records (EHRs). I treat EHR implementation in hospitals as an exogenous treatment to physicians within the hospital and estimate average group time treatment effects on various labor market outcomes. I find that physicians are 10\% more likely to retire, and 4\% more likely to work in an office if they choose not to retire. Finally, for physicians who remain exposed to the technology, the number of patients seen increases. However, claims per patient decreases. These results point to adverse responses when technology is poorly implemented in technical occupations, but also indicates potential efficiency gains for those who do not respond adversely. 
\end{abstract}

\vspace{1.5cm}

\section{Introduction}
How technological innovation affects labor markets is heavily studied, in part because of the vast differences in technology and tasks present even within industries. Yet, there is still much to be learned about modern technology and its interaction with workers. A recent technology roll-out that altered the U.S. health care system dramatically was the electronic health record (EHR). The implementation and use of EHRs were incentivized by the government, leading to a rapid implementation by hospitals and practices nationwide. Since many physicians working in hospitals were exposed to this technology beyond their control, this is an opportune setting to investigate the relationship between rapid technology implementation and labor market decisions for highly skilled workers. 

Specifically, EHRs are computerized medical records stored in advanced systems that may include advanced capabilities such as decision making assistance. EHRs have become increasingly relevant in the U.S. since 2009, when the Health Information Technology for Economic and Clinical Health (HITECH) Act was passed to subsidize hospitals and practices who implement and ``meaningfully'' use an EHR\footnote{This legislation subsidized hospitals which used EHRs “meaningfully”. According to Quatris Healthco, meaningful use standards proceeded in three stages over time. In Stage 1 (2010), MU focused on data capturing and sharing. In Stage 2, which began in late 2012, MU extended to using EHRs for patient incorporation and using the technology as a helper in care. Stage 3 went from 2014-2016 and focused on making data accessible across hospitals (\cite{meanuse})} (\cite{hitech}). President Obama stated in 2009, “To improve the quality of our health care while lowering its cost, we will make the immediate investments necessary to ensure that, within five years, all of America’s medical records are computerized.” (\cite{presquote}). The push for EHR use stemmed from a widespread expectation that EHRs would improve quality of health care while decreasing costs, the gold standard in health policy. For example, a 2005 study estimated hundreds of billions of dollars saved if health information technology were to be fully implemented (\cite{hillestad2005}). The desired movement towards digitization was realized; the percentage of hospitals with the capability of using a basic EHR system went from 9 percent in 2008 to 84 percent in 2015 (\cite{stats}).

Physicians, as the primary users of EHRs, play an important role in whether the expected benefits come to fruition. The daily tasks a physician faces change drastically when a new system is put in place. Some physicians reported that when using EHRs they are less satisfied with their job and have higher stress levels. Senior physicians in particular “loathe the cumbersome, time-consuming data entry that comes with using EHRs.” (\cite{CollierBurnout}). The frustration of using a new technology raises the cost of working in settings that use it, which may lead physicians on the margin to make behavioral changes such as exiting the labor market altogether or shifting towards lower cost work settings. However, the extent to which this frustration actually imposes a meaningful cost is unknown, making physician response an empirical question. Further, the cost imposed is likely to vary based on the type of worker, in particular, age group. Using a difference in difference research design, I estimate the effect of hospital EHR implementation on individual physician labor market choices.

I form a panel of physicians who work in hospitals, hereafter referred to as hospitalists or physicians interchangeably, which spans from 2009-2017. Using CMS Shared Patient Data, I identify hospital-physician pairs that indicate the physician is working within the hospital. Then, I use the AHA Hospital Survey to determine which hospitals implement an EHR, and thus connect hospitalists to EHR use. Finally, I use Medicare Data on Provider Practice and Specialty (MD-PPAS) for detailed information on the patients seen by each hospitalist. For all outcomes, the independent variable of interest is a binary treatment variable capturing exposure to an EHR. I estimate group time treatment effects of EHR exposure on the following physician decisions: (1) retirement, measured based on zero or missing billable activity in all future years of the panel, (2) where to physically work, measured by fraction of patients seen in an office setting and location of the physician office, (3) number of patients seen, and (4) patient and billing activity. 

This paper contributes to the literature's understanding of how health information technology generally affects health care. There is robust empirical research examining the effect of EHRs on patient outcomes and hospital costs, but these studies largely use data prior to the subsidization of meaningful EHR use. Despite a large number of case studies that find generally positive effects (improved patient outcomes and decreased cost) (surveyed in \cite{Buntin2011TheResults}), empirical work has found a mixture of results: that outcomes for median patients do not change as a result of EHRs (\cite{Agha2014TheCare}, \cite{McCullough2016HealthCoordination}, \cite{Meyerhoefer}) while newborns and severe patients experience improvement in health outcomes (\cite{Miller2009}, \cite{Freedman2015}, \cite{McCullough2016HealthCoordination}), and that hospital costs only decrease 6 years after implementation, if at all (\cite{Agha2014TheCare}, \cite{dranove2014trillion}). More relevant to individual physicians' productivity, a number of studies consider the effect of EHRs on productivity in specific settings. One study finds that nursing home productivity increases after adopting health IT (\cite{Hitt2016}), but another finds that physician productivity decreases by 11 percent due to EHRs being adopted in primary-care sites (\cite{Meyerhoefer}).  

I expand this literature, first by considering a different mechanism by which EHRs could affect health care through physicians, which, to my knowledge, has not yet been examined. This is the first empirical study that connects EHR implementation to physician retirement and practice location, and focuses on the difference in response across age groups. Further, I improve on the health information technology literature by considering the primary time period in which EHRs were rapidly implemented. This provides sufficient variation in the timing of implementation and ensures the EHRs are advanced enough to be considered "meaningful" by the government's standards, and are likely affecting physicians' daily life. Similarly to past studies, I consider EHR implementation as a treatment variable in a difference-in-difference framework, but I improve on this by estimating average group time treatment effects for multiple years after implementation, which avoids common problems that arise in typical two way fixed effects estimation of heterogeneous effects, and further provides insight to short vs. long term effects.  

Early research on the inputs to physician retirement decisions found that the most influential factors are personal and financial matters, and that physicians care about their work environment (specifically in the context of managed care organizations), but do not necessarily retire early because of it (\cite{Bahrami2002}). Contrary to this result, I find that EHR implementation led to a 0.006 ppt increase in the likelihood of a senior physician retiring after implementation, a 20\% increase relative to the average retirement for this age group. Physicians less than 60 were 0.002 ppts more likely to retire after exposure, possibly pointing to career switching. These results suggest that EHR implementation may have intensified access to care issues for patients, and that there was a surge of experienced physicians who left the labor force and were no longer available to influence early career physicians. For those who did not retire, I find that young physicians were .014 ppts (4.6\% relative to the mean) more likely to see patients in an office setting the same year as EHR exposure.  Finally, I find a modest, positive effect on patient count completely driven by younger physicians, and a decreasing trend in claims per patient.

Two key assumptions underlie the main analysis. First, I assume that physicians are constrained to use the technology implemented in their work setting. To test this assumption, I investigate the existence of hospital employees with the purpose of using an EHR on behalf of a physician, which I refer to as data assistants. If data assistants are used to ease EHR use for physicians, the results are at least partially driven by their existence. I find that the majority of hiring of data assistants took place in 2013-2014, a two year delay from the vast amount of EHR implementation. Further, my main results are not sensitive to the exclusion of these years, and estimating the effect of EHR implementation only in hospitals without data assistants yields noisy, but similar, results. Second, I assume the decision for hospitals to implement EHRs is exogenous to individual physicians. That is, a physician does not choose their own EHR exposure. While this assumption is reasonable in many settings, there may be instances where a physician is involved in hospital decision making. For this reason, I also consider a sensitivity analysis in which I limit the sample only to hospitals where vertical integration between physician and the hospital is low, as this indicates a low likelihood that EHR implementation is endogenous. The results of the additional analyses indicate that my main findings are not sensitive to endogeneity concerns stemming from joint EHR decision making among physicians and hospitals.







\section{Institutional Details}

\subsection{Background on EHRs}
EHRs have been an important feature of health care since the 1980s. Early in the technology's existence, health care professionals perceived the technology as a complement to paper records, primarily deployed by large academic medical centers to improve efficiency in billing and/or scheduling. Physicians did not interact directly with these early-generation EHRs, and thus were not drastically affected by their implementation. As technological innovation made computers more portable, the usability of EHRs increased, creating what is known as the "physician workspace": a computer station for a physician to interface directly with an EHR to record patient updates. Despite usability, physicians kept the view that EHRs were purely complementary to paper due to burdensome data entry. Automation in data entry was non-existent, making it extremely time consuming for the user. Even when automation was developed for particular machines that performed monitoring, the hospital still held responsibility for the accuracy of information and therefore required physicians to manually check each data entry (\cite{evans2016electronic}). 

The HITECH Act was passed in 2009, designating \$27 billion in government subsidies to entities who used EHRs according to certain guidelines. The guidelines included having at least 80\% of patients in the system, regularly recording answers to specific questions, and protecting the security of the system to ensure privacy. The U.S. allocated subsidies according to stages of "meaningful" use: stage 1 (2011) focused on data collection, stage 2 (2012) extended to using the EHR for care support, and stage 3 (2014) extended to data sharing between practices. This program was successful in  spurring EHR adoption, shown in a 75 percentage point increase in the number of hospitals with EHRs from 2009 to 2015 (\cite{stats}). Figure \ref{fig:meanuse} shows a geographical comparison of U.S. hospitals who have received stage 1 of the meaningful use subsidy in 2011 vs. 2013, revealing the nationwide and rapid expansion of technology. 

\begin{figure}[ht]
    \centering
    \captionsetup{width=.6\linewidth}
    \caption{Hospitals Receiving Meaningful Use Stage 1 Subsidy}
    \includegraphics[scale=.6]{Objects/QS-Hospitals-Receiving-Payments-for-MU-and-Adoption.png}
    \caption*{Source: HealthIT.gov}
    \label{fig:meanuse}
\end{figure}

Figure \ref{fig:EPIC} shows two computer screenshots of what physicians might see in their EHR system. On a daily basis, a physician spends approximately 23.7\%, 17\%, and 15.5\% of their time on documentation, chart review, and inbox management, respectively (\cite{arndt2017tethered}). These are the most time consuming portions of EHR use, according to the same study. In all, if a physician works for 12 hours, they spend roughly 3.2 hours with patients and 5.9 hours interfacing with an EHR. Despite a large amount of time spent with them, physicians still continue to report frustration over EHR use. The most time consuming functions are also reported as the most frustrating (\cite{dymek2021building}). A physician still likely spends almost 2 hours a day managing their inbox: deleting duplicate messages, sifting through messages meant for other members of a care team, and searching relevant information (\cite{dymek2021building}). Another common issue physicians face is the lack of usability of EHR systems, that functions either take minutes to locate or even longer to load. Hospitals often provide training upon adoption, but rapid system upgrades and changes make familiarity with the system difficult. Physicians have been voicing concerns over EHR usability since their beginning, indicating that they are likely responding in their behavior as well.

\begin{figure}[ht]
    \centering
    \captionsetup{width=.4\linewidth}
    \caption{Screenshot of EHR System}
    \includegraphics[scale=.4]{Objects/epic-ehr-screenshot.jpg}
    \includegraphics[scale=.11]{Objects/EHRimage2.jpg}
    \label{fig:EPIC}
\end{figure}


\subsection{EHR and Physician Labor Markets}

This section serves to outline the potential responses of physicians who work primarily in hospitals to EHRs based on the costs and incentives they face. The physician behaviors that I consider in this analysis are closely related and should not be taken as independent. For example, a physician's choice of work setting depends on their previous decision to prolong retirement, and billing activity depends on work setting. Thus, I discuss the behaviors in stages of decision making. 

\subsubsection{Retirement}

Workers in developed countries tend to plan for formal retirement well in advance. Generally, an exogenous shock to a worker's environment is not expected to change retirement age due to the amount of wealth and planning necessary to formally leave the labor force. Classic retirement models structure the decision of retirement by maximizing expected lifetime utility based on future earnings and retirement benefits, or choosing to prolong retirement only if the expected earnings gain from doing so is positive (\cite{gustman1986disaggregated}, \cite{stock1990pension}). However, physicians typically make this decision differently than workers in other industries. Most physicians plan to retire at age 60, but do not actually retire until age 69 (\cite{collier2017challenges}). When the time to retire comes, many physicians hesitate to abandon patients they have seen over the course of their career. The decision to delay retirement is not financial, but altruistic. Thus, retiring may be in a physician's choice set years before the realized decision to leave the labor force. 

It is important to note that age is the most important factor in the decision to retire, implying that a 35 year old physician has very different incentives than a 65 year old physician, and a technology shock will affect the age groups differently. First, I consider physicians already of retirement age when a new technology is implemented. While payment schemes, wealth, and retirement benefits certainly impact the utility a physician receives from working or retiring, I assume the decision to retire is independent of those things since, on average, those in this age group have already prepared financially to retire. Then, a physician will only retire if the utility from working is less than the utility from retiring, where utility from working depends on job satisfaction and personal life factors. Exogenous EHR implementation affects the decision to retire through its affect on job satisfaction. If EHRs impose a burden on daily tasks, there is an inverse relationship between EHR exposure and utility from continuing to work. If EHRs make daily tasks easier, there is a positive relationship. Thus, for a physician indifferent between retiring and continuing to work, EHRs induce retirement if they impose a cost in terms of job satisfaction, or induce continued working life if they impose a benefit to job satisfaction. Media articles focusing on interviews with particular physicians suggest that the implementation of EHRs did in fact lead to retirement in older physicians (\cite{ringel_2019}, \cite{loria_2020}), suggesting a negative impact. 

Formal retirement (leaving the labor force altogether) is not the only way for physicians to transition out of a clinical role. There are opportunities for physicians of any age to switch careers towards administrative roles, teaching, consulting, or hospital management. While it is unlikely that physicians under some age threshold are formally retiring, physicians of any age could be career switching. This decision is impacted similarly to the above discussion regarding formal retirement, where a physician who is indifferent between clinical work and career switching can be affected one way or the other based on how an EHR impacts job satisfaction of current clinical work. As I discuss in the Data Section, I only observe whether a physician stops seeing Medicare patients, not whether they retire or career switch. Thus, the term "retirement" for the purpose of this paper indicates no longer seeing patients, and makes no assertion about what the physician does afterwards. A physician is said to retire in a given year only if it is the first year in which the physician has no future Medicare claims.



\subsubsection{Work Setting}

For most physicians, EHRs will not impose a large enough cost (or benefit) to change the duration of being in the labor market. However, there are other ways physicians might change their behavior under drastic changes to their current working environment. For those who remain practicing in a clinical environment, changing the physical place of work may allow the physician more control over which technologies they use. Generally, physicians may have the ability to change hospitals, move from a hospital to an office, or work in multiple facilities and change the amount of work done in different facilities. I only consider physicians who are classified as a primary care physician and are closely tied to at least one hospital, often referred to as hospitalists or internists (for a detailed discussion of included physicians, see Appendix \ref{app:data}). Some hospitalists work under contract full time for one medical practice, where they have a consistent shift schedule, while others contract themselves out to multiple hospitals, offices, or both. While I do not have sufficient data to observe the type of hospitalist each physician is, I investigate the incentives faces by each type.

First, I consider those who are in a relatively binding contract with the hospital. For hospitalists under full time contract with a medical practice, the physician is more tethered to the hospital. This could impact the response to EHR implementation in that hospital in two ways: (1) through potentially increased use of the EHR in its initial phases, and (2) limited access to other work settings. A typical contract requires 90 to 120 days termination notice and some contracts include do not compete clauses (\cite{yasgur_by_-_yasgur_2016}), making a shift in work setting costly or excluding it from the choice set. Since these hospitalists are established in their hospital, they work a significant amount of time there. When an EHR is implemented, they may become familiar with it more quickly and be less likely to desire to switch work settings. Thus, I classify these physicians as "sticky", since they are not able to change behavior in the short run.

Alternatively, there may be hospitalists that work in multiple hospitals or work in both hospitals and offices throughout the sample period. If one hospital implements a costly (or beneficial) EHR, these workers can more easily shift all or some of their work towards (or away from) different facilities to allocate to the place where they gain the most job satisfaction. As this type of hospitalist is less sticky, if there is any effect of EHR exposure on work setting it would be among this type. 


\subsubsection{Productivity}

There are likely still physicians who are not induced to change their behavior on the basis of retirement or work setting due to EHR implementation. A lack of response could be due to 4 factors: the hospitalist is bound to their current work environment, EHRs make no impact on the work environment, EHRs are costly, but not costly enough to shift, or EHRs improve job satisfaction in the current work environment. For physicians who make no physical changes to work environment after EHR exposure, it is natural to assess whether EHRs affect output in current work environment. An important aspect of the implementation of EHRs is whether or not they allow users to be more productive. A main purpose of the HITECH Act was to improve efficiency of care, specifically to decrease time spent on administrative burden. 

Whether EHRs affect productivity in terms of patients seen is theoretically ambiguous. If they serve their purpose to decrease administrative burden, the physician would be freed up to see more patients. In this case, I expect productivity to increase. However, if utilizing an EHR actually imposes an additional administrative burden as suggested in physician interviews, the number of patients could either decrease or stay the same, indicating a decrease in productivity.  

Further, there is a unique relationship between EHR use and efficiency in terms of billing activity done by the physician. Billing activity refers to the number of claims filed per patient seen. EHRs may improve efficiency if they remove the need for repeated or excess testing. However, if having an EHR makes it easy for a physician to claim for additional items with the click of a button, the adverse incentive may lead physicians to claim for more items per patient. Thus, both productivity and efficiency are empirical questions. 




\section{Data}

Using various linked data sets, I construct a physician-level panel spanning from 2009-2017 which measures physicians' exposure to EHRs over time, outcomes related to physician practice location and billable activity, and other relevant physician, hospital, and practice characteristics. I describe the different data sets used to construct the panel below. I include a detailed outline of data and variable construction in Appendix \ref{app:data}.

\subsection{MD-PPAS}

The Medicare Data on Provider Practice and Specialty (MD-PPAS) is a database of physicians which details claim counts, physician specialty, and practice location from 2009-2017. Included in the MD-PPAS is physician NPI, primary and secondary specialties, both patient and total billing counts in up to 12 different zip codes, and fraction of patients in different settings such as office, inpatient or emergency room. I first limit the sample of physicians based on specialty type, since my analysis relies on EHR implementation in hospitals. I only include physicians who reported their primary specialty as hospitalist (physician who self-identifies as hospitalist or has 90\% of patients in inpatient setting), internal medicine, general practice or family practice in at least one year. However, I exclude physicians who list themselves as a specialist in all but one year. Since the focus of the analysis is physicians working in hospitals, I also drop physicians with less than 70\% of patients in a hospital setting in all years, and I drop physicians with insufficient claim counts to show indications of behavior.

This data includes information on physicians which I use to construct the dependent variables for my analysis. Generally, these include the decision to retire (leave clinical work), the decision to switch to a different work setting, number of patients seen, and overall billing activity. I describe these outcome variables and their construction in detail in Section \ref{sec:outcome}.


\subsection{Shared Patients}

CMS Shared Patient data records annual information on count of Medicare patients and frequency of billing associated with the same patient across providers within a given time period (30, 60, 90, or 180 days). These data are available from 2009 to 2015. For example, if a primary care physician refers a patient to a specialist, then those two physicians have that particular shared patient in common. The number of shared patients are collected in 30, 90, or 180 day intervals. I focus on the 30 day interval, and I employ this data to detect physicians who work closely with hospitals. I limit the entities by tax-code to only include shared patients for physician-hospital pairs, where the physicians in these pairs match the physicians from MD-PPAS. I am specifically interested in primary care physicians who have a close working relationship with hospitals, who do rounds within at least one hospital. The types of physicians are already limited to primary care, but there still may be office-based primary care physicians in the shared patient data. Therefore, I limit the sample of pairs based on a threshold of same day patients which depends on the number of years the physician appears in the data. I only include physician-hospital pairs who have non-missing data with at least 30 shared patients per year, which gives a threshold of work relationship without excluding physicians who have no claims for a number of years.

\subsection{AHA Survey Data}

Using hospital NPI, I link the physician-hospital pairs from the shared patient data to the American Hospital Association (AHA) survey, which contains information on hospital-level EHR use and other characteristics. I record the first year a physician is exposed to an EHR based on implementation in the hospitals they share patients with. I then aggregate this data to the physician level. Since the Shared Patient data does not extend to 2017 as the MD-PPAS data does, in the analysis I drop physicians who are not treated by 2015 as I do not observe whether they become treated in 2016 or 2017. That is, the data contains no units which are "never-treated".

\subsection{Physician Compare}

Finally, I include a physician's graduation year from Physician Compare. I limit to physicians who graduated before 2005, as anyone who graduated medical school after that will be finishing residency during the span of the data and will exhibit labor market changing behavior which could be correlated to EHR exposure purely because of switching work settings during a time of rapid EHR implementation.

\subsection{Outcome Variables}\label{sec:outcome}

The first dependent variable I consider is the decision for a physician to stop seeing Medicare patients, which I loosely call retirement. This variable is constructed as follows: 
$$\text{retire}_{it}=\mathbbm{1}\{FC_{it}=0 \text{ and retire}_{i,t-1}=0\}, $$
where $FC_{it}=\sum\limits_{j=t+1}^T\text{(claim count)}_{ij}$ is all future claims for physician $i$ in year $t$. Retire is therefore an indicator variable set to one in the first year that a physician has no claim counts at any point in the future. \footnote{Alternatively, I could also define retirement using future number of patients. I use claim count for a more conservative measure of retirement, but using patient count yields identical results. In Appendix \ref{app:years}, I limit the sample years to 2009-2015 to assess the robustness of the main retirement definition.}

Next, I consider whether physicians exhibit behavior that suggests they have shifted setting or location of work. I measure this in the data in three ways. First, I consider the fraction of patients a physician sees in an office-based setting, which is directly available in the data. Second, I construct an indicator variable equal to one if a physician sees a positive fraction of patients in an office-based setting. This binary measure of practice setting offers less variation but captures a more discrete change in which physicians enter or leave the office-based setting entirely. Third, based on the physician's practice zip code, I form an indicator variable equal to one if the physician's list of zip codes changes from one year to the next. While no single variable directly measures physician movement across hospitals, these three outcomes collectively provide insight into whether physicians desire to shift away from or towards EHRs. Finally, I consider outcome variables which explore inputs to overall efficiency in health care, the number of patients seen and claims filed per patient.


\subsection{Summary Statistics}

I show summary statistics for the entire sample in Table \ref{tab:sumstats}. Only 3\% of physicians (approximately 770) retire over the course of the panel. 30\% of the physician-year observations indicate working in an office in some capacity, with about a tenth of the number of patients being seen in an office setting. By construction, all physicians are exposed to an EHR at some point in the sample, so the variation in treatment variables comes from the timing of exposure. Physicians are typically exposed about one fourth of the way through the sample, but there is a positive number of physicians first exposed in each year from 2009-2015. The average physician is 45 years old, works with 1.5 hospitals and 1.3 systems. 
\import{Table Code}{overall stats.tex}


I also include a table of means comparing physicians younger than 60, at least 60, and physicians who ever retire (who can fall in either age category) in Table \ref{tab:splitstats}. Senior physicians see more patients and work in more hospitals on average than younger physicians. These older physicians also see a larger fraction of patients in an office setting and are more likely to work in an office in general. Finally, physicians in the higher age bracket are less likely to switch zip codes. The age groups are exposed to EHRs in similar ways.


\import{Table Code}{split stats.tex}


In Figure \ref{fig:treatmentgraph}, I present a graph of the variation in treatment over time. In 2009, 28\% of the sample of physicians were already exposed to an EHR. That is, three quarters of the sample had no affiliation with EHRs at the beginning of the sample period. Since I drop any physicians who do not have exposure to an EHR by 2015, 100\% of the sample is exposed by then. The black line represents the average fraction of a physician's hospitals which are utilizing EHRs over time. This shows that, along with becoming first exposed over the sample period, physicians are also seeing EHRs implemented in more of their own hospitals over time.

\begin{figure}[t]
\centering
\captionsetup{width=.45\linewidth}
    \caption{Treatment Variables Over Time}
    \includegraphics[scale=.5]{Objects/sum_stats_year.pdf}
    \label{fig:treatmentgraph}
\end{figure}



\section{Empirical Strategy}\label{sec:empstrat}

For each outcome variable described in Section \ref{sec:outcome}, I treat EHR implementation as an exogenous treatment variable, and I seek to estimate average treatment effects of EHR exposure. In this setting, treatment effects are heterogeneous across physicians who are exposed in different years since the underlying characteristics of hospitals who adopt in different years may be correlated with the decision to adopt. Therefore, to avoid negative weighting issues this causes in a classical two way fixed effects event study specification, I instead estimate average treatment effects for a specific group $g$ at time $t$: 
$$ATT(g,t)=\mathbbm{E}[Y_t(g)-Y_t(0)|G_g=1],$$
where $G_g=1$ for those in group $g$. A group indicates all physicians treated, or first exposed to an EHR, in the same year. To estimate the heterogeneous treatment effects, I employ the doubly robust estimator established in \citeauthor{sant2020doubly} (\citeyear{sant2020doubly}). Other estimators that similarly address the concerns yield similar results, presented in Appendix \ref{app:estimators}. I also present classical two way fixed effects results with physician and year fixed effects in Appendix \ref{app:estimators}. These results yield similar patterns with slightly smaller magnitudes. 

Once I estimate the $ATT(g,t)$ for each group and year, I aggregate the estimates over groups to a more familiar event study plot. For select outcomes, I present the original de-aggregated estimates to show that results are not driven by a particular year of treatment.  Further, a weighted average of estimates is taken to provide a single $ATT$ value for each outcome; these values are presented in the notes of each table. Simultaneous confidence bands, calculated using bootstrapped standard errors, are presented. 

\subsection{Assumptions}

There are several assumptions necessary to identify the parameters of interest, $ATT(g,t)$. First, I assume that treatment is not reversed once it occurs. That is, once a physician is exposed to an EHR, they cannot be un-exposed. For physicians that do not switch hospitals, this assumption is supported by both the institutional details of hospital EHR implementation and the data itself. An EHR is a costly technology and requires a significant amount of collaboration to implement. A hospital does not have incentive to un-implement an EHR once it is implemented. They may add features or switch vendors, but do not reverse EHR use. This institutional detail is supported by Figure \ref{fig:hosp_treat}, which shows EHR status of a 2\% random sample of hospitals over time.\footnote{Note that out of the 81 hospitals included in the random sample, 4 of them show a reversal of treatment status. Upon further inspection, each of these hospitals show such a reversal due to a discrepancy in the coding of whether the hospital uses an EHR partially or fully. Those who use the EHR partially are coded as not fully implementing an EHR since it is not clear whether physicians in the sample would be exposed yet or not. Hospitals with this discretion are coded as implementing an EHR in the first year that they answered "fully" to the survey question.} Further, since we think of treatment at the physician level as exposure to the technology, this assumption is satisfied when thought of as whether the physician has ever used an EHR. Once they are exposed to the EHR for the first time, that experience cannot be forgotten. However, when considering the outcomes of patient count and billing activity, it could be that a physician moves from a hospital that uses an EHR to a hospital that does not, affecting the number of claims observed in the data. This is why I limit the sample of physicians for these outcomes to those who remain with the same set of hospitals throughout the entire period. I discuss this limitation to the data further in Section \ref{sec:patientcount}.

\begin{figure}[ht!]
    \centering
    \captionsetup{width=.7\linewidth}
    \caption{Hospital EHR Adoption Over Time}
    \includegraphics[scale=.5]{Objects/hosp_treat.pdf}
    \label{fig:hosp_treat}
\end{figure}

Second, I assume physicians do not anticipate EHR exposure prior to occurrence. Since the technology has different vendors and varies in capabilities, anticipation may be a concern if physicians learn that the system will be implemented and then do not actually use it until a future period. Generally, even a complex EHR system can be completely set up within a year, and most systems take 6 to 9 months (\cite{uzialko_2021}). Therefore, I proceed in the main specification assuming no anticipation. However, I explore and present results for a one year anticipation period in Appendix \ref{app:anticipation}. 

As is usual in a difference-in-differences framework, I assume a version of parallel trends based on not-yet-treated units. I assume that, conditional on years of experience, average outcomes for those treated in group $g$ would have followed a parallel trend as those in groups treated in later periods. This assumption would be violated under external conditions in certain years that may be correlated with labor markets, such as a major recession. The time period I study, 2009-2017, is reassuringly stable. In my analysis, I present p-values for a Wald test of pre-trends. In most cases, there is no evidence of a pre-trend. Further, I investigate pre-trends extensively in Appendix \ref{sec:pretrends}. 

To interpret these estimators as the causal effect of EHR exposure on various labor market outcomes, there are additional institutional assumptions necessary. First, I assume that when a physician works in an electronic record utilizing-hospital, they utilize the electronic health record system fully. That is, there are no ways for the physician to continue practicing without using the EHR. I investigate this assumption in Section \ref{sec:dataass}. Second, I assume that a physician's exposure to EHRs is exogenous. That is, the physician does not influence the hospital's decision to implement an EHR. I investigate this assumption in Section \ref{sec:endogeneity}. 





\section{Effect of EHR Exposure on Labor Market Outcomes}


\subsection{Retirement}


In Figure \ref{fig:retirefirst}, I present aggregated group time treatment effects of being exposed to an EHR on the likelihood of retirement for the full sample of physicians, as well as split samples for those at least 60 (typical retirement age) and less than 60 years of age. These effects are a weighted average of the $ATT(g,t)$ parameters discussed in Section \ref{sec:empstrat}. The estimates in the top panel suggest that being exposed to an EHR leads to a .003 ppt increase in the likelihood of retiring in the first year after exposure, and a .0045 ppt increase in the second year after exposure. While numerically small, these effects are statistically and economically meaningful relative to the proportion of physicians who retire in the sample, .03. That is, the effect represents a 10\% (or 15\% in the second year) increase in the likelihood of retirement relative to the mean. Even though there is no visual indication of a violation of the parallel trends assumption in the pre-periods, the p-value of a Wald test for pre-trends indicates some evidence for such a violation. Therefore, in Appendix \ref{sec:pretrends}, I present confidence intervals robust to various specifications of a violation in parallel trends. The results are largely similar to those presented here.

\begin{figure}[ht]
    \centering
    \captionsetup{width=.85\linewidth}
    \caption{Effect of EHR Exposure on Retirement}
    \includegraphics[scale=.6]{Objects/retire_plot.pdf}
    \label{fig:retirefirst}
    \vspace{2mm}
    \caption*{\footnotesize{\textit{Notes:} The top panel shows average group time treatment effects aggregated over groups to an event study plot. The bottom show these results for different subgroups of physicians by age. The p-value listed for each graph corresponds to a Wald test for pre-trends. Confidence intervals shown are simultaneous confidence bands accounting for multiple hypothesis testing. Overall ATT for all, $<$ 60, $>=$ 60 with SE in parentheses: 0.003 (0.0008), 0.002 (0.0009), 0.005 (0.002), respectively.}}
\end{figure}


Next, I examine how the results differ when considering physicians in different age brackets: pre-retirement ($<= 60$) and typical retirement age ($> 60$). These results are shown in the bottom panel of Figure \ref{fig:retirefirst}, and suggest that senior physicians are driving the positive estimate found above in the year after exposure, while younger physicians are driving the result for the second year after exposure. A physician who is at least 60 years old is .01 ppts more likely to retire after being exposed to an EHR, a 25\% increase relative to the mean. In comparison, physicians less than 60 are far less likely, if at all, to respond in the first year after EHR exposure. Physicians less than 60 are approximately 16\% more likely to retire two years after exposure. There are two potential reasons for the different response time. First, if younger physicians are leaving clinical work for administrative or consulting roles, the preparation time for this type of switch (preparing a resume, applying for new jobs, etc.) could be longer than one year, whereas formal retirement can be decided and executed within one year. Alternatively, physicians who are not yet retirement age may attempt to learn complex EHRs for a longer period of time before deciding the cost is too high. 

One argument against this result is that physicians are not leaving a clinical environment, they are simply choosing to avoid seeing Medicare patients. While EHR use should not be correlated specifically with Medicare patients, there may be an unobserved event leading physicians to stop seeing Medicare patients that aligns with EHR implementation. The Center for Medicare and Medicaid services publishes a list of physicians who have officially opted out of seeing Medicare patients. None of the physicians in my sample have selected into this option. Further, the differential timing of retirement from older vs. younger physicians is reassuring, as any Medicare change should not affect physicians of different ages differently.  



\subsection{Place of Work}

There is still a large portion of physicians not induced to retire due to EHR exposure. Thus, I consider the decision to switch work settings using multiple outcomes. First, I investigate the extent to which hospitalists work in an office and whether this changes due to EHR exposure. I use variation in two different variables: (1) an indicator variable equal to 1 if the physician has any positive number of patients in an office in a given year, and (2) the fraction of patients a physician sees in an office in a given year. Second, I investigate whether EHR exposure causes a change in the likelihood of switching zip codes. This analysis does not include individuals who retire at any point in the sample, as a zero could indicate dropping out of the data instead of working solely in a hospital. 

I first discuss the effect of EHR exposure on the probability of working in an office setting, shown in Figure \ref{fig:officefirst}. The top panel shows aggregated group time effects for physicians of any age, and suggests that EHR exposure increases the likelihood of working in an office in the same year as exposure by .014 ppts, a 4.6\% increase relative to the mean. This result indicates that hospitalists are more likely to start working in an office in the year they are exposed to an EHR, though the effect is not as large as the one found for physicians retiring. Splitting the sample by age reveals that this result is driven by the younger sample of physicians. For retirement age physicians, the estimates become very noisy and I do not draw any conclusions. Younger physicians, however, show evidence of switching behavior. The .015 ppt increase in the likelihood of retiring is equivalent to a 5.4\% increase relative to the mean for this age group. While there is not strong evidence for the existence of a pre-trend, I consider the robustness of these results to various violations in the parallel trends assumption in Appendix \ref{sec:pretrends}. This analysis yields the same result.


\begin{figure}[ht]
    \centering
    \captionsetup{width=.85\linewidth}
    \caption{Effect of EHR Exposure on Likelihood of Working in Office}
    \includegraphics[scale=.6]{Objects/officeind_plot.pdf}
    \label{fig:officefirst}
    \vspace{2mm}
    \caption*{\footnotesize{\textit{Notes:} The top panel shows average group time treatment effects aggregated over groups to an event study plot. The bottom show these results for different subgroups of physicians by age. The p-value listed for each graph corresponds to a Wald test for pre-trends. Confidence intervals shown are simultaneous confidence bands accounting for multiple hypothesis testing. Overall ATT for all, $<$ 60, $>=$ 60 with SE in parentheses: -0.0004 (0.008), -0.004 (0.008), 0.02 (0.02), respectively.}}
\end{figure}

The previous results is driven by physicians who are not already working in an office prior to EHR exposure, but it could also be the case that physicians shift the amount of work they do in an office. Thus, I estimate the effect of EHR exposure on the fraction of patients seen in office based settings.\footnote{Since this variable is also tied to the total number of patients seen, I also estimate this analysis with the outcome of total number of patients seen in an office, and the results are the same.} I show the estimates and 95\% confidence intervals for this outcome in Figure \ref{fig:officesecond}, where I find no evidence that EHR exposure affects the fraction of patients seen in office settings. This does not seem to be a margin where physicians are changing their behavior due to EHR use.


\begin{figure}[ht]
    \centering
    \captionsetup{width=.85\linewidth}
    \caption{Effect of EHR Exposure on Fraction of Patients Seen in Office}
    \includegraphics[scale=.6]{Objects/officefrac__plot.pdf}
    \label{fig:officesecond}
    \vspace{2mm}
    \caption*{\footnotesize{\textit{Notes:} The top panel shows average group time treatment effects aggregated over groups to an event study plot. The bottom show these results for different subgroups of physicians by age. The p-value listed for each graph corresponds to a Wald test for pre-trends. Confidence intervals shown are simultaneous confidence bands accounting for multiple hypothesis testing. Overall ATT for all, $<$ 60, $>=$ 60 with SE in parentheses: -0.0003 (0.004), -0.0007 (0.005), -0.006 (0.018), respectively.}}
\end{figure}

The previous results only capture switching behavior which involves office based settings. Another way to examine physician switching behavior is to investigate whether the zip codes in which the physicians work is changing, which captures both office and other switching behaviors. Again, I exclude physicians who ever retire. The outcome variable is equal to 1 if their list of primary zip codes changes. This variable is limited in how broadly it captures switching behavior because multiple scenarios are flagged as changing zip codes: leaving a particular zip code, adding one new zip code but leaving the rest unchanged, or completely changing zip codes. While this is a vague description of behavior, it provides a different insight into changes occurring because of EHR implementation that can be combined with the previous outcome variables to draw conclusions about general behavior.

The effects of EHR exposure on changing zip codes over time are shown in Figure \ref{fig:zip}. For the full sample of hospitalists, there is a statistically positive effect observed in each year after EHR exposure, but there is evidence that this result is driven by a pre-trend before exposure. Therefore, I do not make any causal arguments about this effect. However, I include a plot of confidence intervals that are robust to specified violations of parallel trends for each age group in Appendix \ref{sec:pretrends}. These results reveal that under linear violations of the parallel trends assumption, there is still a positive relationship between EHR exposure and changing zip codes for younger physicians. When taken together, these estimates suggest there is slight evidence of physicians who do not retire changing aspects of their work environment, but not on a large scale, indicating stickiness in place of work. 

\begin{figure}[ht]
    \centering
    \captionsetup{width=.85\linewidth}
    \caption{Effect of EHR Exposure on Likelihood of Changing Zip Codes}
    \includegraphics[scale=.6]{Objects/zip_plot.pdf}
    \label{fig:zip}
    \vspace{2mm}
    \caption*{\footnotesize{\textit{Notes:} The top panel shows average group time treatment effects aggregated over groups to an event study plot. The bottom show these results for different subgroups of physicians by age. The p-value listed for each graph corresponds to a Wald test for pre-trends. Confidence intervals shown are simultaneous confidence bands accounting for multiple hypothesis testing. Overall ATT for all, $<$ 60, $>=$ 60 with SE in parentheses: 0.045 (0.006), 0.046 (0.007), 0.042 (0.018), respectively.}}
\end{figure}




\subsection{Patient Count and Billing Activity}\label{sec:patientcount}

As a measure of productivity or efficiency, I now estimate the effect of EHR exposure on patient count and billing activity per patient. If the sample includes physicians who shift place of work, this may lead to biased estimates since a physician could switch hospitals due to an EHR and then change practice behavior due to the change in environment, not due to the EHR. Thus, I limit the sample to only physicians who work with the same one hospital through the entire sample. These hospitalists do not retire or see patients in another hospital. Thus, when an EHR is implemented, the physician remains utilizing the EHR for the remainder of the sample. 

The effect of EHR exposure on patient count is shown in Figure \ref{fig:patient}. For hospitalists of any age, patient count increases by 17.5 patients in the year of EHR exposure, 34.7 patients in the year after exposure, 16.6 patients in the second year after exposure, and then the effect disappears. These estimates translate to a 5-10\% increase in the number of patients seen relative to the mean. Interestingly, when splitting the sample by age group, the effect becomes null for those of retirement age, suggesting that technology use impacts productivity dependent on worker type (in this case, age). Again, I present a sensitivity analysis considering the parallel trends assumption in Appendix \ref{sec:pretrends}. 

\begin{figure}[ht]
    \centering
    \captionsetup{width=.85\linewidth}
    \caption{Effect of EHR Exposure on Patient Count}
    \includegraphics[scale=.6]{Objects/patient_plot.pdf}
    \label{fig:patient}
    \vspace{2mm}
    \caption*{\footnotesize{\textit{Notes:} The top panel shows average group time treatment effects aggregated over groups to an event study plot. The bottom show these results for different subgroups of physicians by age. The p-value listed for each graph corresponds to a Wald test for pre-trends. Confidence intervals shown are simultaneous confidence bands accounting for multiple hypothesis testing. Overall ATT for all, $<$ 60, $>=$ 60 with SE in parentheses: 14.1 (4.6), 15.8 (5.4), -0.29 (13.0), respectively.}}
\end{figure}

While this paper is interested the supply side factors of patients seen, this variable is also driven by demand for health care. A major change in the U.S. health care system, the Affordable Care Act (ACA) was passed in 2010 and took effect in 2014 for the majority of states that partook. If my results are driven primarily by the group of hospitals who implemented EHRs in 2014, one might be concerned that the legislation may be driving the increased patient count. The ACA had potentially opposing effects on Medicare patients: Medicare patients may be crowded out by an increased demand in Medicaid patients, or the small changes to Medicare may have led to an increase in utilization from the Medicare population. A recent study found no negative spillovers to the Medicare population due to the ACA (\cite{carey2020impact}), suggesting that demand change is not driving my results. Further, I present average group time treatment effects for each treated group in Figure \ref{fig:patientgroup}. Each group shows the same pattern of a small increase in patient count in the first 0-2 years after implementation. 

\begin{figure}[ht!]
    \centering
    \captionsetup{width=.6\linewidth}
    \caption{Effect of EHR Exposure on Patient Count by Group}
    \includegraphics[scale=.47]{Objects/patient_group.pdf}
    \label{fig:patientgroup}
    \vspace{2mm}
    \caption*{\footnotesize{\textit{Notes:} Plot shows average group time treatment effects for each treatment group.}}
\end{figure}

Since I found in the earlier stages of analysis that physicians may have shifted away from hospitals due to EHR exposure, I also investigate whether this increase in the number of patients seen is due to the decrease in working physicians. For example, if a hospital loses a portion of their physicians, they may divide those patients between the remaining physicians. In the long run, the hospital should hire new physicians to replace those that left, but if this does not happen right away then these estimates could be driven by the retiring physicians instead of EHR use. In Figure \ref{fig:numphys}, I present a graph showing the relationship between EHR adoption and the number of physicians working in the hospital. While these are not causal estimates, they do show that the trend seems to be more physicians working in hospitals, and there is no apparent shock in the short run after EHR implementation. This suggests that the increase in patient count is not due to the exit of physicians observed in the earlier analysis. 

\begin{figure}[ht]
    \centering
    \captionsetup{width=.85\linewidth}
    \caption{Effect of EHR Exposure on Number of Physicians in Hospitals}
    \includegraphics[scale=.4]{Objects/numberofphysicians_plot.pdf}
    \label{fig:numphys}
\end{figure}

Next, I consider whether EHR exposure affects claims filed per patient. On average, claim counts are roughly 4 times the number of patients seen in a year. Shown in Figure \ref{fig:claim}, there is a decreasing trend in claims per patient in the three years after EHR exposure. Two years after exposure, claims per patient decreases by .25, roughly 6\% of the average. This indicates a potentially advantageous effect of EHRs, that excessive billing is reduced. Though noisier for retirement age physicians, this effect is seen in both age groups.    

\begin{figure}[ht]
    \centering
    \captionsetup{width=.85\linewidth}
    \caption{Effect of EHR Exposure on Claims per Patient}
    \includegraphics[scale=.6]{Objects/claim_per_patient_plot.pdf}
    \label{fig:claim}
    \vspace{2mm}
    \caption*{\footnotesize{\textit{Notes:} The top panel shows average group time treatment effects aggregated over groups to an event study plot. The bottom show these results for different subgroups of physicians by age. The p-value listed for each graph corresponds to a Wald test for pre-trends. Confidence intervals shown are simultaneous confidence bands accounting for multiple hypothesis testing. Overall ATT for all, $<$ 60, $>=$ 60 with SE in parentheses: 4.04 (33.89), 25.77 (31.26), -161.06 (133.35), respectively.}}
\end{figure}


\subsection{Late Adopters as Comparison Group}

For the results presented, if EHR exposure affects physician behavior, the effect typically occurs in the first 0-2 years after exposure. Almost all specifications point to the effect of EHR exposure becoming null in the 3-4 years after exposure. One reason for this may be the composition of the hospitals which adopt at different times. Consider event time $t+4$. The physician included in the estimation for this coefficient are those exposed in 2010 compared with those who did not adopt until 2015, implying that the underlying hospitals implemented EHRs in either 2010 or 2015. It is likely that the group of hospitals who adopted in 2015 are compositionally different than early adopters, especially if poorer hospitals adopted later. For this reason, I have focused discussion of results on the immediate years after exposure when hospitals the comparison groups are more comprehensive. 

\begin{figure}
    \centering
    \includegraphics[scale=.5]{Objects/control_histogram.pdf}
    \caption{Caption}
    \label{fig:my_label}
\end{figure}


\section{Physician Influence of EHR Decision in Hospitals}

\subsection{Hiring Data Assistants}\label{sec:dataass}

The effects found in the above analysis could be biased if there is an external event occurring that is correlated with EHR implementation and exposure. The existence or hiring of employees with the purpose of utilizing electronic health records on a physician's behalf is a way for physicians to continue working in a hospital with an EHR but not experience exposure to the technology. Employees of this type are traditionally referred to as scribes, and are assigned a medical tax ID when working in hospitals. they have the following official titles in tax data: Coding Specialist (Hospital Based), Health Information Technologist and Registered Record Administrator. I will refer to any employee in these categories as a data assistant. In this Section, I investigate whether data assistants were hired in accordance with EHR implementation and whether the presence of data assistants affects physician response to EHRs. This is particularly relevant for the productivity and efficiency outcomes where I find that EHRs improve both outcomes.  

First, using NPPES data on all NPIs and their tax information, I analyze the existence of data assistants in hospitals. Figure \ref{fig:dataassistant_histogram} presents a frequency plot of the year of activation for every tax code that falls in the categories listed above. The total number of these registered employees is 875, and the first data assistants ever registered were in 2005. From 2005 to 2013, 15-60 more employees were registered each year. The graph is clearly skewed towards later years, where a significant increase in the number of new data assistants occurred in 2014. If hiring data assistants was directly correlated with both EHR implementation and physician frustration, one would expect the increase to occur from 2011-2012, when a majority of physicians first became exposed to EHRs. The results of my main specification are not sensitive to leaving out later years, which is a good indication that the results are not driven by the enumeration of data assistants. 

\begin{figure}[t]
\centering
\captionsetup{width=.5\linewidth}
\caption{Frequency of Data Assistant Enumeration by Year}
\includegraphics[scale=.5]{Objects/dataassistant_histogram.pdf}
\label{fig:dataassistant_histogram}
\vspace{2mm}
    \caption*{\footnotesize{\textit{Notes:} Histtogram uses NPPES data on the enumeration of certain types of NPIs to estimate the number of new data assistants being hired in each year of my sample.}}
\end{figure}

Further, I investigate the occurrence of hospitals and data assistants sharing patients in the CMS Shared Patient data. I create a list of hospital-years for which the hospital has a positive number of shared patients with a data assistant. I merge this to the physician-hospital pairs and create a physician level variable for whether any hospital worked with is associated with a data assistant. I find that .1\% of physician-year observations are associated with data assistants. This sample of physicians is too small to analyze whether data assistants are more likely with EHR exposure, yielding largely noisy estimates. However, I limit the sample to only physicians not associated with data assistants and find that the magnitude of estimates are similar, albeit more noisy. Details of this analysis are given in Appendix \ref{app:DA}. 

\subsection{Endogeneity of EHR Implementation}\label{sec:endogeneity}

Another way physicians may avoid EHR exposure is to exhibit decision making power to either delay or encourage EHR implementation in hospitals. This would make the binary treatment variable endogenous, biasing the results of the main specification, as it includes physicians who are selecting into their own exposure. After the sample period ends, the government began penalizing hospitals for not using health information technology meaningfully, so implementation eventually occurs across the board. However, during the sample, physicians may impact the timing or occurrence of implementation. I address this question by limiting the sample of hospitals in a meaningful way, only including those hospitals which are considered to have low vertical integration with physicians. That is, physicians are employees and less likely to be decision makers. I then redefine exposure as when a physician is exposed to an EHR in low integration settings. I present details of how I define integration and results for this specification in Appendix \ref{sec:endog}. In general, group time estimates are similar to those in the main specification, with the exception of the probability of working in an office setting, which may indicate some correlation between EHR adoption in high integration settings and acquisition of physician offices. 


\section{Conclusion}

In this study, I analyze the effect of implementing a now-common technology in hospitals, electronic health records, on various physician labor market outcomes. This technology has been controversial and debated as a burdensome load on physicians that may contribute to physician burnout, and therefore patient outcomes. Understanding how physicians respond to this technology is important as physician labor markets have implications for access to care, quality of care, and costs. 

I utilize data on shared patients, hospital EHR use, and physician billing activity. Using a difference in difference framework, I find that physicians of all ages who work in hospitals are more likely to stop seeing Medicare patients altogether when exposed to an EHR, where those of retirement age have an even larger increase relative to younger physicians. Further, I find slight evidence that physicians who do not retire change where they work because of EHR exposure. Finally, physicians who utilize EHRs exhibit an increase in the number of patients seen and a decrease in claims filed per patient. 

The implications of these results are threefold. First, a surge of physicians left clinical settings in a concentrated number of years due to EHR exposure, potentially worsening access to care. In a field where younger physicians learn a lot by watching those more experienced than them, this particular loss of senior physicians may have long term negative effects on patient outcomes, but this is a question left to future research. Further, physicians are exhibiting behavior that suggests they incur switching costs to avoid external burdens. When policymakers consider regulations on physician practice it is important to understand that physicians may take measures to avoid such regulation. Finally, these results speak to the ongoing debate of whether EHRs are generally beneficial to its users. While the technology may be a burden, physicians are able to see more patients because of their use, and do not seem to over-bill because of them. 

This paper contributes to our understanding of health information technology by providing empirical evidence of the theoretically ambiguous physician response to electronic health records being rapidly implemented in hospitals. This contribution is relevant to EHR vendors, hospitals, and policymakers who are involved in further implementation of complex technology. 




\clearpage

\renewcommand*{\bibfont}{\footnotesize}

\printbibliography

\clearpage


\appendix

\section{Data}\label{app:data}

This section details the process of creating the data set used in the analysis of the main paper. 

\subsection{Physician Specialties by Tax Code}\label{sec:taxcode}

I begin by connecting every National Provider Identifier (NPI)\footnote{Data can be downloaded from \hyperlink{https://download.cms.gov/nppes/NPI/Files.html}{\text{https://download.cms.gov/nppes/NPI-Files.html}}} to a description of their tax code\footnote{Data can be downloaded from \hyperlink{https://nucc.org/index.php/code-sets-mainmenu-41/provider-taxonomy-mainmenu-40/pdf-mainmenu-53}{\text{https://nucc.org/index.php/code-sets-mainmenu-41/provider-taxonomy-mainmenu-40/pdf-mainmenu-53}}}. I then categorize NPIs by key words in their tax code description. Any description containing ``Internal Medicine", ``Hospitalist", ``Family Medicine", or ``General Practice" is classified as a primary care physician (PCP), and any description containing ``hospital" is classified as a hospital. I create two data sets, one containing only PCPs and one containing only hospitals. These data are saved to use in identifying the relevant NPIs in the CMS Shared Patient Data. 



\subsection{CMS Shared Patient Data}\label{sec:sharedpat}

For years 2009-2015, the CMS collected detailed information on the number of Medicare patients shared between any two NPIs within 30, 90, or 180 day intervals\footnote{Data can be dowloaded from \hyperlink{https://www.nber.org/research/data/physician-shared-patient-patterns-data}{https://www.nber.org/research/data/physician-shared-patient-patterns-data}}. I use data that captures shared patient activity in 30 day intervals. I merge the shared patient data to the filtered tax code data created in Section \ref{sec:taxcode} to identify NPIs who are either PCPs or hospitals. I filter the pairs to include one physician and one hospital; there are 12.6 million of these pairs. Some pairs are duplicates from the hospital being listed first or second in the shared patient data. I combine duplicates into one observation, summing the same day count variable. Once duplicated are removed, there are 7.1 million observations. Most of the pairs have very few shared patients, which is not indicative of the physician working inside the hospital. I drop any pairs who do not have at least 30 shared patients in the same day per year the pair appears in the data.\footnote{As a sensitivity analysis, I also consider thresholds of 10 and 60. Results are similar and can be found in Figure \ref{sec:chart}.} The final list of pairs consists of 1.3 million observations. 

\subsection{Pair-Level Variables}

I combine the physician-hospital pairs created in Section \ref{sec:sharedpat} with various data sets for hospital or physician level information. First, I merge to CMS Physician Compare\footnote{Data can be downloaded from \hyperlink{https://data.cms.gov/provider-data/dataset/mj5m-pzi6}{https://data.cms.gov/provider-data/dataset/mj5m-pzi6}} for information on each physician's graduation. This data contains more information on physician quality, but I am limited to time-invariant information since Physician Compare spans 2012-2015 and 2010-2012 is a time of major EHR implementation. I drop any physicians who graduated medical school after 2004, since graduating after that means leaving residency or graduating during the span of my main data, 2009-2017, and potentially exhibiting labor market changes that seem associated with EHRs but are not. 

For hospital level variables, I use an AHA-NPI crosswalk to merge the pairs to the Annual Hospital Administration Survey (AHA Survey) from 2009-2015. The variables I include from this data are a hospital's number of beds, organization type, system ID, and EHR use. There are 4,253 unique AHA hospitals in the shared patient data. I drop any hospitals that aren't in the AHA Survey due to lack of information on EHR use. After limiting the hospitals, there are 780,000 observations of pair-years left. Next, I investigate the hospitals with missing information for EHR use. There are 89 hospitals in the data who never answer the survey question about EHR use; I drop these hospitals. If a hospital does not answer the survey question in one year, but the year before and after have an identical survey answer, I fill in the missing year of information. Then there are 4,253 observations with missing data for the EHR survey question. I make a further limitation to hospitals with at least 10 beds. There are only 58 unique hospitals in the data with less than 10 beds, and these are likely very different from the rest of the sample of hospitals. Then, I create a physician-level variable which captures the average number of beds in the hospitals they work with in each year. I sum a physician's same day count over their hospitals in a given year to create a physician level patient count variable. I also create this variable only for hospitals who are using an EHR and only for hospitals who are not using an EHR.

I use the general survey question about each hospital's EHR use to define my key treatment variable. The survey asks the extent to which the hospital utilizes an EHR and allows for three answers: not at all, partially utilize an EHR, and fully utilize an EHR. I create a binary variable equal to one if the answer is fully utilize, zero otherwise. The data will be aggregated up to the physician level due to physician level outcome variables, so I first transform the pair-level EHR variable into a treatment variable at the physician level. I create a variable capturing the first year that at least one of a physician's connected hospitals uses an EHR. For the endogeneity discussion in Section \ref{sec:endog}, I create a similar variable but only count a physician as exposed if a hospital categorized as "low-integration" adopts an EHR. The level of vertical integration between hospital and physician is defined as in \citeauthor{dynan1998assessing} (\citeyear{dynan1998assessing}) using the organization type of the hospital. Hospitals classified as IPA or PHO are low-integration, and any other type is high0integration. I also create a variable that captures the fraction of hospitals a physician works with that use an EHR, which varies on the dimension of EHR use and on number of hospitals worked with. Finally, the last variable I create before aggregating to the physician level is an indicator for whether the physician works with the same set of hospitals through the entire sample. I create this variable by comparing a physician's number of hospitals in each year to the maximum number of hospitals they ever work with. If there is any year in which the a physician's number of hospitals is less than the maximum number of hospitals they ever work with, the indicator for never having a new NPI is set to 0. I use this variable to limit the sample in Section \ref{sec:patientcount}. 

Finally, I aggregate to the physician level by only keeping variables that do not vary by hospital. The variables in the physician level data are year, physician NPI, graduation year, average number of beds in hospitals worked with, years of experience, number of hospitals, fraction of hospitals with EHR, minimum year exposed to EHR, minimum year exposed to EHR in low integration hospital, number of systems, and never works with a new NPI. I complete this data to include years 2016 and 2017, but leave time varying variables missing for those years. Thus, this is a balanced panel of physicians. I save this data to be merged with physician labor market activity in Section \ref{sec:appmdppas}.

\subsection{Physician Labor Market Activity}\label{sec:appmdppas}

Using physician NPI, I merge the physician treatment data to Medicare Data on Provider Practice and Specialty (MD-PPAS)\footnote{Information on data located at \hyperlink{https://resdac.org/cms-data/files/md-ppas}{https://resdac.org/cms-data/files/md-ppas}}, which spans 2009-2017. This data contains variables on physician specialty, Medicare claim counts in various zip codes, unique number of patients seen, fraction of patients seen in specific settings, patient demographics, and physician date of birth. Once the data is merged, I make a further limitation to drop any physicians with less than 70\% of their total patients in a hospital setting.\footnote{I consider different thresholds as a sensitivity analysis. The results are similar, and can be found in Section \ref{sec:chart}} This limitation is to continue ensuring that I am focusing on physicians working in hospitals who will be subject to their EHR use. With this limitation, there are 214,000 observations left in the data. 

Next, I create the dependent variables used in the analysis. The first outcome is whether a physician chooses to retire or not. First, I sum a physicians claim counts across zip codes into one variable for total claim count in a given year. Then, I create a variable that sums up a physicians claims in all future years. In 2009, this variable sums up all claims from 2010-2017, and so on. Then I create a variable for the first year that a physician has a future claim count of zero. However, this counts retirement one year too early. For example, if a physician has no claims from 2014 to 2017, the minimum year variable is set to 2013, but the first year of retirement is 2014. Thus, I add one year to the minimum year variable. I then define the outcome variable, retire, as one if a physician ever retires and the year is equal to the first year of retirement. I utilize a physician level variable for whether they ever retire for sample limitations in analyzing other outcomes. 

The next outcome variables consider level of labor market activity in office settings. The first decision to make is how to handle years where a physician has missing data for claims. A year of missing data can mean the physician had no claims that year, or the observation was removed due to insufficient claims (less than 11). A conservative assumption is to assume a year of missing data means zero claims, thus, I set all missing claim counts to zero. The data already contains a variable for the fraction of patients seen in an office setting, which is used in the analysis, but I further create an indicator variable for whether the physician sees any positive amount of patients in an office setting. 

I also create an indicator variable for whether the physician switches zip codes from one year to the next. The physicians in this data work in up to 12 different zip codes, so I their zip codes into one string variable, a list of all zip codes in a year. I make this variable into wide format: zip list 2009, zip list 2010, ... , zip list 2017.  In year $t$, I define a variable change zip that is equal to one if the zip list in $t-1$ is different from the zip list in $t$, zero otherwise. Finally, the variables for unique patient count and claim count are already defined. This is the final data used in the paper, with approximately 215,000 observations. 



\section{Sensitivity Analysis}

\subsection{Analyzing Pre-Trends Assumptions}\label{sec:pretrends}

Work by \citeauthor{rambachan2019honest} (\citeyear{rambachan2019honest}) reveals the need to carefully assess the parallel trends assumptions commonly used in difference in difference designs. In my main specification I assume that, conditional on covariates, average outcomes for those treated in group g would have followed a parallel trend as those in groups treated in later periods. I assess this assumption by testing whether the coefficients showing the effect of EHR exposure on an outcome prior to actual treatment are jointly zero. All main specification graphs show p-values from this test. In most analyses, I fail to reject the null that there does not exist a pre-trend. Yet, in some cases, I reject this null. This section utilizes the tools established by \citeauthor{rambachan2019honest} (\citeyear{rambachan2019honest}) to demonstrate robustness of the results found in the main specification, and to investigate whether an effect exists in the cases where a pre-trend may be driving results. 

\subsubsection{Retirement} 

First, I consider the estimated effect of EHR exposure on retirement. While visually there does not appear to be a strong pre-trend, for the full sample of physicians and physicians $< 60$, the p-value indicates a pre-trend may exist. Since the positive effect seems to be a possibility in event time 1 and event time 2, I present a plot of robust confidence intervals under various possible assumptions of parallel trends violations for these periods in Figure \ref{fig:pre_retire}. The original confidence interval is shown in yellow, and in gray are confidence intervals robust to varying functions of parallel trend violations. When allowing for linear trend violations (M=0), the effect of EHR exposure on retirement is still positive. As I allow for more non-linearity (M$>$0) in these variations, this effect becomes zero. This is true for both event time 1 and event time 2. 

\begin{figure}[ht]
    \centering
    \captionsetup{width=.5\linewidth}
    \caption{}
    \includegraphics[scale=.5]{Objects/retire_pretrends_plot.pdf}
    \label{fig:pre_retire}
    \vspace{2mm}
    \caption*{\footnotesize{\textit{Notes:}}}
\end{figure}

\subsubsection{Work in Office Likelihood}

Next, I examine pre-trends for the outcome variable indicating whether the physician does any work in an office. The p-values do not indicate pre-trends for the full sample or for physicians $< 60$, but I reject no pre-trends for physicians of retirement age. As with retirement as the outcome, I show robust confidence intervals in Figure \ref{fig:pre_work}. However, the main results show up in event time 0 and 1, so I show confidence intervals for these periods. Further, I show these for the full sample of physicians, but also for older physicians separately since the pre-trend seems to differ slightly. When allowing for linear pre-trend violations, I find very similar results to the main specification. For physicians $>= 60$, there is a higher chance that the effect of EHR exposure on the probability of working in an office is positive. This would be theoretically intuitive since older physicians may struggle with the technology more than younger physicians and choose to switch settings. However, I do not make strong conclusions given the null results found in most sensitivity analyses. 

\begin{figure}[ht]
    \centering
    \captionsetup{width=.5\linewidth}
    \caption{}
    \includegraphics[scale=.5]{Objects/work_pretrends_plot.pdf}
    \label{fig:pre_work}
    \vspace{2mm}
    \caption*{\footnotesize{\textit{Notes:}}}
\end{figure}

\subsubsection{Fraction of Patients in Office}

The next outcome considered is the fraction of patients seen in an office setting. All p-values suggest there are no pre-trends. I show robust confidence intervals for the full sample of physicians in the first and second year of exposure in Figure \ref{fig:pre_frac}. These results hold no matter the assumption on the violation of pre-trends. 

\begin{figure}[ht]
    \centering
    \captionsetup{width=.5\linewidth}
    \caption{}
    \includegraphics[scale=.5]{Objects/frac_pretrends_plot.pdf}
    \label{fig:pre_frac}
    \vspace{2mm}
    \caption*{\footnotesize{\textit{Notes:}}}
\end{figure}

\subsubsection{Change Zip Codes Likelihood}

I now consider pre-trends for the effect of EHR exposure on whether the physician changes zip codes. In all main specification results, there is indication of a clear pre-trend and thus it is not clear whether there is a positive effect due to EHR exposure or due to this trend occurring prior to exposure. Thus, I present confidence intervals that are robust to different specifications of parallel trend violations for each age group at event time 0 and event time 1 in Figure \ref{fig:pre_zip}. At event time 0, it seems younger physicians are more likely to change zip codes after EHR exposure under a linear pre-trend violation and slightly non-linear pre-trend violations, although the magnitude of the effect is likely smaller than the main specification shows. This is also true for younger physicians in event time 1. For physicians of retirement age, it seems the entire effect was driven by a pre-trend and thus there is no effect of EHR exposure on this group's likelihood of changing zip codes. 

\begin{figure}[ht]
    \centering
    \captionsetup{width=.5\linewidth}
    \caption{}
    \includegraphics[scale=.5]{Objects/zip_pretrends_plot.pdf}
    \label{fig:pre_zip}
    \vspace{2mm}
    \caption*{\footnotesize{\textit{Notes:}}}
\end{figure}

\subsubsection{Patient Count}

Now I examine the sensitivity of the main specification results where the outcome considered is the number of patients seen. None of the p-values indicate a violation of parallel trends, but I present a plot of confidence intervals robust to specified variations in parallel trends as a robustness check in Figure \ref{fig:pre_patient}. These plots indicate that, even under nonlinear deviations in parallel trends, there is a positive effect of EHR exposure on patient count. Under a linear variation in parallel trends, the effect is positive, yet the magnitude is nearly half of the magnitude found in the main specification. 

\begin{figure}[ht]
    \centering
    \captionsetup{width=.5\linewidth}
    \caption{}
    \includegraphics[scale=.5]{Objects/patient_pretrends_plot.pdf}
    \label{fig:pre_patient}
    \vspace{2mm}
    \caption*{\footnotesize{\textit{Notes:}}}
\end{figure}

\subsubsection{Claim Count}

Finally, I assess the results found in the main specification for the outcome variable claim count. The main specification indicates a downward trend in the years leading up to exposure and a positive effect in the year of exposure and the year after exposure for younger physicians, and no effect for older physicians. In Figure \ref{fig:pre_claim}, I show confidence intervals robust to violations in parallel trends for these two age groups. For physicians $>= 60$, this plot shows a null effect just as in the main specification. For physicians $< 60$, there still seems to be a positive effect on both event years, although under exactly linear variations in parallel trends, this effect is smaller or zero.  

\begin{figure}[ht]
    \centering
    \captionsetup{width=.5\linewidth}
    \caption{}
    \includegraphics[scale=.5]{Objects/claim_pretrends_plot.pdf}
    \label{fig:pre_claim}
    \vspace{2mm}
    \caption*{\footnotesize{\textit{Notes:}}}
\end{figure}

\subsection{Various Changes to Specification}\label{sec:changes}

For each outcome, there are various adjustments I make to the main specification in order to show robustness. Some of these changes address potential endogeneity concerns, while some provide evidence that the results do not depend on thresholds which may be arbitrary. I will outline each of these specification adjustments and then present specification charts which account for different combinations of these changes. 

\subsubsection{Physician Selection into EHRs}\label{sec:endog}

The main specification of the paper defines a binary treatment variable for whether a physician is exposed to an EHR in any hospital they work in. These results are biased if physicians endogenously select whether they are exposed to EHRs, particularly by participating in the hospital's decision to implement. The results do not indicate that this is a huge issue on average, since we would expect physicians not to respond on the margin of retirement or switching work setting if they are choosing to be exposed to EHRs. Especially in larger hospitals, this is unlikely to occur as an individual physician will not have much input to high level decisions. However, as an attempt to provide more evidence of labor market changes as a result of EHR exposure, I provide results for specifications with an alternate binary treatment variable: whether a physician has been exposed to an EHR in a hospital considered to have low vertical integration with physicians. 

I define a hospital's level of integration based on organizational structure, where independent practice associations (IPAs) and physician-hospital organizations (PHOs) are classified as low vertical integration and other organization types are classified as high vertical integration (\cite{dynan1998assessing}). 



\subsubsection{Anticipation}\label{app:anticipation}

While there is evidence to suggest the time it takes to implement an EHR does not exceed one year, there still may be some anticipation if physicians are expecting implementation in the hospitals they work in. Thus, the specification charts that I present in Section \ref{sec:chart} include specifications where one year of anticipation occurs. 

\subsubsection{Limiting Sample Years}\label{app:years}

In the main sample, I define dependent variables using years 2009 to 2017 and drop any physicians who were not exposed to an EHR by 2015 due to data constraints. This leads to estimation based on not yet treated units, because all never treated units are dropped. As a robustness check, I limit sample years to 2009-2015 and use never treated units as the control group. 

\subsubsection{Hiring Data Assistants}

As discussed in the body of the paper, I address potential concerns that the hiring of new employees may be driving the results for productivity and efficiency. Thus, as another specification, I restrict the data to only include physicians who are exposed in hospitals that do not record employing any data assistants. 

\subsubsection{Data Thresholds}

There are two main thresholds I define in the data process that were arbitrarily chosen. Thus, I present alternate specifications that adjust these thresholds. First, when deciding which hospitals a physician is working closely with, I drop any pairs that fall below 30 shared patients per year. The goal of this threshold is to eliminate pairs for which the physician does not have close ties to the hospital. In alternate specifications, I lower the threshold to 10 patients per year and raise the threshold to 60 patients per year. Second, I only want to include hospitalists in the analysis, and thus drop any physicians that fall below 70\% of patients in an inpatient setting. In alternate specifications, I change this threshold to 50\% and 90\%. While these are still arbitrary, they provide some evidence that changing the threshold does not dramatically change the results. 


\subsection{Specification Charts}\label{sec:chart}

I present overall average treatment effects for each outcome variable under different combinations of the specification adjustments discussed in Section \ref{sec:changes}. On each chart, the main specification is labeled in green and the average confidence bands across all combinations is shown as an orange band. 

\subsubsection{Retirement}

The specification chart for the retirement outcome is presented in Figure \ref{fig:retire_chart}. The main specification yields an average treatment effect of .0025 (significant at all levels). This coefficient is on the higher end of coefficients from all combinations of specifications. However, almost all of the coefficients are positive. Some noisy estimates are to be expected when there are multiple limitations placed on the data. The average confidence interval for all combinations is (.001,.003), so the main specification lies in this interval. These results are strongly in favor of there being a positive effect of EHR exposure on retirement. 

\begin{figure}[ht]
    \centering
    \caption{Outcome: Retire}
    \includegraphics[scale=.6]{Objects/retire_chart.pdf}
    \label{fig:retire_chart}
\end{figure}

\subsubsection{Change Work Setting}

The specification charts for outcomes work in office, fraction of patients seen in office, and changing zip codes are presented in Figures \ref{fig:work_chart}, \ref{fig:fracoffice_chart}, and \ref{fig:zip_chart}, respectively. First, I discuss the overall average treatment effect of EHR exposure on the likelihood of working in an office. While the main specification showed a slight increase in likelihood in the year of exposure, the overall treatment effect is 0. Most of the specifications agree with the main specification, although more coefficients fall below 0 than above 0. The average confidence interval is approximately (-.016, .001), encompassing the main specification.  

\begin{figure}[ht]
    \centering
    \caption{Outcome: Work in Office}
    \includegraphics[scale=.6]{Objects/office_chart.pdf}
    \label{fig:work_chart}
\end{figure}

Second, I discuss the results for whether physicians change the fraction of patients seen in an office setting due to EHR exposure. The main specification

\begin{figure}[ht]
    \centering
    \caption{Outcome: Fraction of Patients in Office}
    \includegraphics[scale=.6]{Objects/office_frac_chart.pdf}
    \label{fig:fracoffice_chart}
\end{figure}

Finally, I discuss the specification chart regarding a physician's likelihood of changing zip codes due to EHR exposure. These results should be taken in conjunction with Section \ref{sec:pretrends}, which analyzes the extent to which this result is driven by a pre-trend. The main specification yields an average treatment effect of .04 (significant at all levels), which is in the average confidence interval across specifications. One thing to notice in this graph is that there are two changes to specification for which the resulting average treatment effect is grouped towards one end of the distribution or the other. Specifically, when the percent of patients seen in an inpatient setting is 90\%, the coefficient is likely to be higher than when it is 50\% or 70\%. That is, a more strict definition of hospitalist yields a higher estimate for the likelihood of changing zip codes (although the variance of estimates is not particularly high). Further, using the main EHR definition tends to yield lower estimates than the low integration definition. That is, endogeneity is biasing this estimate slightly down. 

\begin{figure}[ht]
    \centering
    \caption{Outcome: Change Zip Codes}
    \includegraphics[scale=.6]{Objects/change_zip_chart.pdf}
    \label{fig:zip_chart}
\end{figure}




\subsubsection{Patient Count and Claims per Patient}

\begin{figure}[ht]
    \centering
    \caption{Outcome: Patient Count}
    \includegraphics[scale=.6]{Objects/patientcount_chart.pdf}
    \label{fig:pat_chart}
\end{figure}

\begin{figure}[ht]
    \centering
    \caption{Outcome: Claims per Patient}
    \includegraphics[scale=.6]{Objects/claim_per_patient_chart.pdf}
    \label{fig:cpp_chart}
\end{figure}



\subsection{Alternative Estimators}\label{app:estimators}

There is a robust recent literature considering the issues with estimating heterogeneous treatment effects with a two way fixed effects specification, the classic approach in staggered treatment setups. To avoid these problems, in my main specification I use the average group time effect estimator established in \citeauthor{callaway2021difference} (\citeyear{callaway2021difference}). In this section, I discuss other potential estimators and why I ultimately included average group time treatment effects as the main specification. 

Before I present these estimators, I present results from a two way fixed effects setup that may suffer bias from negative weighting. Specifically, I estimate the following equation:
$$\text{outcome}_{i,t}=\sum_{j=-4}^{-2} \text{rel\_year}_{j} + \sum_{j=0}^{4} \text{rel\_year}_{j} + \delta_i + \gamma_t,$$
where outcome$_{i,t}$ is one of the six outcomes discussed previously, rel\_year$_j$ is an indicator equal to 1 if year $t$ is $j$ years relative to expansion, $\delta_i$ are physician fixed effects, and $\gamma_t$ are year fixed effects.
These results from each outcome are presented in Figure \ref{fig:twfe}. These results show similar patterns to the main specification, with slight variations in magnitude. 

\begin{figure}
    \centering
    \captionsetup{width=.8\linewidth}
    \caption{Results: Two Way Fixed Effects}
    \includegraphics[scale=.8]{Objects/twfe_plot.pdf}
    \label{fig:twfe}
    \vspace{2mm}
    \caption*{\footnotesize{\textit{Notes:} Event study plots from two way fixed effects estimation are shown for each outcome variable.}}
\end{figure}

One method that addresses issues with the TWFE specification by differencing out fixed effects is established in \citeauthor{gardner2021two} (\citeyear{gardner2021two}). This approach is commonly known as two stage difference in differences and is very clever. However, the approach requires a group of never-treated units as the comparison group, a drawback for my study in that I remove all never-treated units to utilize all years of MD-PPAS data and because never-treated hospitals are likely compositionally different from those treated in the majority of the sample. Nevertheless, I limit the years of data to 2009-2015 and present estimates of this estimator in Figure \ref{fig:esimators}. Generally, the results have the same sign with a slightly larger magnitude, driven by the inclusion of never-treated units. 

\begin{figure}[ht]
    \centering
    \captionsetup{width=.57\linewidth}
    \caption{Other Event Study Estimators}
    \includegraphics[scale=.57]{Objects/estimators_graph.pdf}
    \label{fig:esimators}
    \vspace{2mm}
    \caption*{\footnotesize{\textit{Notes:} Results from various methods of estimation are shown, along with 95\% confidence intervals.}}
\end{figure}

Another estimation method established first used by \citeauthor{cengiz2019effect} (\citeyear{cengiz2019effect}) is commonly known as stacked regression. In this method, one re-frames event study data into groups of sub-experiments that are then stacked on top of each other, defining a group of treated units and "clean" control units. The drawback of this method is choosing an event study window that must be the same for each treatment group, where a large window eliminates many treated units and a small window does not allow for estimation over many years. I present the results using this method with an event window of one year in Figure \ref{fig:esimators}. The results are very similar to the main specification, with a smaller magnitude in some cases. 





\section{Heterogeneity Analysis}

The main analysis focuses on the differential effects of technology for older vs. younger physicians. However, another important question is whether physicians in rural vs. urban settings respond differently to the technology since the implications for behavior changes are high in areas that already suffer from access to care issues. Thus, I present average treatment effects for each outcome where the sample is split between physicians in rural and urban areas. Further, to investigate the same idea, I present results for physicians who typically work at small hospitals vs. those who typically work at large hospitals. These average treatment effects are shown in Figure \ref{fig:1}. For most outcomes, the results do not differ along these dimensions. The only outcome for which the estimate changes sign is the indicator for whether the physician works in an office setting. Physicians in an urban area are less likely to work in an office after EHR exposure, while rural physicians are more likely to work in an office after EHR exposure, but the sign switch does not hold when looking at the hospital size parallel. This results is interesting and may be driven by differing physician-hospital relationships. I leave further investigation to future research. 


\begin{figure}[t!]
\begin{subfigure}{0.48\textwidth}
\caption{Retire}
\includegraphics[width=\linewidth]{Objects/retire_heterog.pdf}
\label{fig:a}
\end{subfigure}\hspace*{\fill}
\begin{subfigure}{0.48\textwidth}
\caption{Fraction of Patients in Office}
\includegraphics[width=\linewidth]{Objects/frac_office_heterog.pdf}
\label{fig:b}
\end{subfigure}

\medskip
\begin{subfigure}{0.48\textwidth}
\caption{Indicator for Office }
\includegraphics[width=\linewidth]{Objects/office_heterog.pdf}
 \label{fig:c}
\end{subfigure}\hspace*{\fill}
\begin{subfigure}{0.48\textwidth}
\caption{Change Zip Code} 
\includegraphics[width=\linewidth]{Objects/change_zip_heterog.pdf}
\label{fig:d}
\end{subfigure}

\medskip
\begin{subfigure}{0.48\textwidth}
\caption{Patient Count} 
\includegraphics[width=\linewidth]{Objects/patient_heterog.pdf}
\label{fig:e}
\end{subfigure}\hspace*{\fill}
\begin{subfigure}{0.48\textwidth}
\caption{Claim per Patient}
\includegraphics[width=\linewidth]{Objects/claim_per_patient_heterog.pdf}
 \label{fig:f}
\end{subfigure}

\caption{Heterogeneity Analysis} \label{fig:1}
\end{figure}





















\end{document}