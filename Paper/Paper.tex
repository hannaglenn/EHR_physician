\documentclass[11pt]{article}
\usepackage[utf8]{inputenc}
\usepackage{amsmath,amssymb,hyperref,array,xcolor,multicol,verbatim,mathpazo}
\usepackage[normalem]{ulem}
\usepackage[pdftex]{graphicx}
\usepackage{fullpage}
\usepackage{import}
\usepackage{adjustbox}

\usepackage[backend=biber,style=authoryear,
sorting=ynt,citestyle=authoryear]{biblatex}
\addbibresource{papercitations.bib}
\usepackage{setspace}
\onehalfspacing
\addtolength{\skip\footins}{2pc plus 5pt}

\title{Labor Markets and Technological Change: Evidence from Electronic Health Records}
\author{Hanna Glenn}
%\date{\today}

\DeclareLabeldate[online]{%
  \field{date}
  \field{year}
  \field{eventdate}
  \field{origdate}
  \field{urldate}
}

\begin{document}

\maketitle



\vspace{1.5cm}

The relationship between technology and labor markets is a longstanding topic of interest in economics and in policy making. This relationship is relevant in many aspects: technology causing displacement, whether low-skilled vs. high-skilled workers are differentially impacted, or the productivity gains that can be realized due to technology. In many capacities, across differing countries and industries, people tend to care about how technology affects their jobs and their children's future jobs. This paper seeks to understand this relationship under a major technology change in the U.S. healthcare system, the implementation of electronic health care record systems. 

Electronic Health Records (EHRs) have become increasingly relevant in the U.S. since 2008, in part after former President Obama stated in 2009, “To improve the quality of our health care while lowering its cost, we will make the immediate investments necessary to ensure that, within five years, all of America’s medical records are computerized.” (\cite{presquote}). The movement towards digitization in health care was expected to have immediate and substantial impacts on cost and quality of care. In particular, a 2005 study estimated hundreds of billions of dollars saved if health information technology were to be fully implemented (\cite{hillestad2005}). Improvement in quality of care was expected due to physicians having more time to spend with patients, better communication across different physicians on the same care plan, and decision-making assistance to avoid harmful mistakes. Such expectations on cost and quality led policymakers to incentivize the use of EHRs in hospitals with the the Health Information Technology for Economic and Clinical Health (HITECH) Act in 2008 (\cite{hitech}). This legislation subsidized hospitals which used EHRs “meaningfully”\footnote{According to Quatris Healthco, meaningful use standards proceeded in three stages over time. In Stage 1 (2010), MU focused on data capturing and sharing. In Stage 2, which began in late 2012, MU extended to using EHRs for patient incorporation and using the technology as a helper in care. Stage 3 went from 2014-2016 and focused on making data accessible across hospitals (\cite{meanuse})}. The movement towards EHR use is evidenced in hospitals from 2008 onward; the percentage of hospitals with the capability of using a basic EHR system went from 9 percent in 2008 to 84 percent in 2015 (\cite{stats}).

Specifically, an EHR is digital version of a patient’s medical records. These include detailed accounts and notes of medical history and can become advanced enough to provide the physician suggestions for care. Physicians are the primary users of this technology and have a significant role in determining whether potential benefits of EHRs are realized. However, the day-to-day life of physicians changed drastically with the implementation of EHRs. What was expected to be a helpful tool in general has turned out to be burdensome and frustrating for the persons using it. In individual interviews, multiple physicians reported that when using EHRs they are less satisfied with their job and have higher burnout and higher stress levels. Senior physicians in particular were found to “loathe the cumbersome, time-consuming data entry that comes with using EHRs.” (\cite{CollierBurnout}). A physician with the choice to stop working or move to a smaller private practice may choose to do so when the cost of using an EHR in a hospital becomes too high. This paper seeks to understand whether the implementation of EHRs in hospitals led to changes in physician labor market behaviors: where to work, productivity levels, and retirement.

Using CMS Shared Patient Data from 2009-2015, I construct a panel of physician-hospital pairs that captures a specific working relationship between the pair, most likely the physician working in the hospital. I link the hospitals to AHA survey data for information on EHR use and then aggregate the data to the physician level. This physician-level data gives information on an individuals' level of exposure to electronic health records. To measure the working decisions of physicians, I utilize (1) physician level information, also from the shared patient data, that captures total working life (referrals to labs, pharmacies, other physicians, etc), and (2) an office-based physician level data set to observe whether physicians are moving from hospitals to private practice. For my main analysis, I use an event study estimation where treatment is defined as a physicians' exposure to an EHR at any of the hospitals they work in. A key identifying assumption in this analysis is that hospitals choose to implement EHRs not based on any factors that are also correlated with physician working decisions.  

\textcolor{red}{Paragraph about findings}

This paper contributes to our understanding of two main ideas: the broad effects of technology on labor markets, and the more specific effects of EHR implementation in various healthcare dimensions.

This setting is an important application to the broad understanding of how technology impacts labor for various reasons. First, health is a major expense in the United States, accounting for approximately 17 percent of GDP. The potential cost savings of EHRs are extremely important to understand given their potential to also improve patient satisfaction, as policymakers are continuously interested in achieving these two things simultaneously. Further, there has been discussion in both economics and medical literature about whether we have a physician shortage and, similarly, access to care issues in the U.S. (\cite{cooper2002economic}). Thus it is important to understand whether electronic health records are driving physicians out of the market or decreasing the amount of patients that can be seen by one physician, both of which could potentially worsen these issues. \textcolor{red}{This paragraph needs significant editing}

The effects of EHR implementation on costs and quality of healthcare prior to 2010 have been studied extensively. Despite many case studies and hospital-level analysis that indicate large positive effects on health (\cite{Buntin2011TheResults}), economic research has showed that the technology has only improved health outcomes for patients with severe conditions, but have not led to improvements for the median patient (\cite{Agha2014TheCare}; \cite{McCullough2016HealthCoordination}; \cite{Meyerhoefer}). Further, there have been no significant cost decreases due to EHRs in the short or long run. If any cost reductions are realized at all, it is not until at least 6 years after implementation (\cite{dranove2014trillion}). However, it may be that it takes more time to see the potential cost reductions realized, so there may still be potential benefits realized in the future. Finally, there have been preliminary case studies and interviews seeking to understand physicians' opinions on EHRs. These have drastically different findings depending on the setting studied, where there is evidence of older physicians having more frustration regarding the technology than younger physicians. This paper's contribution to the space is threefold: (1) this is the first study to my knowledge that empirically analyzes the relationship between physician labor markets and EHRs, (2) since there seems to be a disconnect between the potentials of EHR technologies and the realized effects, this paper will speak to a potential mechanism that is causing this puzzle, and (3) since these studies are all considering time periods before 2014, they do not capture the full period of EHR implementation.


\section{Background}

\section{Data}

Using various linked datasets, I construct a physician-level panel spanning from 2009-2015 which measures physicians' exposure to EHRs over time and other relevant characteristics. The data used to construct this panel are described in detail below.

The main data I utilize is the CMS Shared Patients Data. For a pair of National Provider Identifiers (NPIs), this data records the number of patients who bill both of the entities under Medicare in the same day. For example, if a Medicare patient is referred to a specialist by a primary care physician, then those two entities have a shared patient in common. I limit the entities to only include the number of shared patients of physician-hospital pairs. I care specifically about physicians who have a close working relationship with hospitals, preferably those who do rounds within at least one hospital. To achieve this, I limit the qualifications for a physician-hospital pair in my data. The data only includes doctors who have a primary care type of role, excluding any specialists. The reason for this is that a primary care physician who does rounds in a hospital likely takes on a hospitalist role and does not do much office work. I further limit to those pairs which have a substantial number of patients billed together in the same day. This is to avoid including the potential physician who works in an office but happens to send a small number of patients to a nearby hospital. Substantial in this case is defined as having at least 30 patients bill both the physician and the hospital in the same day in at least one year, \textcolor{red}{where the exact number chosen is explored in the appendix}. With these limitations, the physicians left in the data are likely doing a number of rounds physically inside hospitals, since it is unlikely that many patients will see a primary care doctor in a private office and then immediately go to a hospital in the same day. 

Using hospital NPI, I link the physician-hospital pairs to the American Hospital Association survey, which contains information on hospital-level EHR use and other characteristics. I then aggregate to the physician level, where physician EHR use is measured as the percentage of their hospitals which use an EHR in a given year. Additional physician-level characteristics, such as years of experience, come from Physician Compare. 

Summary statistics from the data constructed are shown below. \textcolor{red}{Summary stats table} \textcolor{red}{Talk about what the summary stats show here}

Figure 1 shows the adoption of EHRs by hospitals and the exposure to EHRs by physicians over time. \textcolor{red}{talk about graph here} \textcolor{red}{put graph here}



\section{Retirement}

\section{Work Setting}

\section{Physician Productivity}





\end{document}