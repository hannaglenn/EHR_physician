\documentclass[11pt]{article}
\usepackage[utf8]{inputenc}
\usepackage{amsmath,amssymb,hyperref,array,xcolor,multicol,verbatim,mathpazo}
\usepackage[normalem]{ulem}
\usepackage[pdftex]{graphicx}
\usepackage{fullpage}
\usepackage{import}
\usepackage{adjustbox}
\usepackage{booktabs}
\usepackage{bbm}
\usepackage[font=normalsize,labelfont=bf]{caption}
%\captionsetup{justification=raggedright,singlelinecheck=false}


\usepackage[backend=biber,style=authoryear,
sorting=ynt,citestyle=authoryear]{biblatex}
\addbibresource{papercitations.bib}
\usepackage{setspace}
%\singlespacing
\doublespacing
\addtolength{\skip\footins}{2pc plus 5pt}

\title{Labor Markets and Technological Change: Evidence from Electronic Health Records}
\author{Hanna Glenn}
%\date{\today}

\DeclareLabeldate[online]{%
  \field{date}
  \field{year}
  \field{eventdate}
  \field{origdate}
  \field{urldate}
}

\begin{document}

\maketitle



\vspace{1.5cm}

\section{Introduction}

How technological innovation affects labor markets is often controversial, in part because of the vast differences in technology and tasks present even within industries. For example, industrial robots decreased the share of hours worked in low-skilled workers from 1993-2007, but increased overall labor productivity  (\cite{graetz2018robots}), and even low-skilled workers in Norway who were not displaced by robots, but saw other workers displaced, experienced lower job satisfaction due to fear of displacement (\cite{schwabe2020automation}). Moreover, shocks to an existing labor market could have downstream effects on customers in that market. A prime instance of this is in health care, where technological innovation in health care may contribute to physician burnout, which has been shown to affect patient satisfaction (\cite{shanafelt2002burnout}), or technological change may encourage physician retirement, possibly limiting patient access to care and ultimately affecting patient health. In this paper, I investigate physician labor market changes as a result of a major technology shift in the U.S. health care system, the implementation of electronic health records.

Electronic Health Records (EHRs) are computerized medical records which include detailed accounts and notes of medical history, stored in advanced systems with additional capabilities such as providing suggestions for care. EHRs have have become increasingly relevant in the U.S. since 2009, when the Health Information Technology for Economic and Clinical Health (HITECH) Act was passed to subsidize hospitals and practices who implement and ``meaningfully'' use an EHR\footnote{This legislation subsidized hospitals which used EHRs “meaningfully”According to Quatris Healthco, meaningful use standards proceeded in three stages over time. In Stage 1 (2010), MU focused on data capturing and sharing. In Stage 2, which began in late 2012, MU extended to using EHRs for patient incorporation and using the technology as a helper in care. Stage 3 went from 2014-2016 and focused on making data accessible across hospitals (\cite{meanuse})} (\cite{hitech}). President Obama stated in 2009, “To improve the quality of our health care while lowering its cost, we will make the immediate investments necessary to ensure that, within five years, all of America’s medical records are computerized.” (\cite{presquote}). Researchers and government officials expected the movement towards digitization to affect efficiency in health care substantially. A 2005 study estimated hundreds of billions of dollars saved if health information technology were to be fully implemented (\cite{hillestad2005}). The percentage of hospitals with the capability of using a basic EHR system went from 9 percent in 2008 to 84 percent in 2015 (\cite{stats}), revealing the nationwide movement towards digitization.

Physicians, as primary users of this technology, play an important role in whether potential benefits of EHRs are realized. With the rapid implementation of a complex technology, the practical tasks that take place in a physician's daily life changed just as rapidly. In interviews, physicians reported that when using EHRs they are less satisfied with their job and have higher stress levels. Senior physicians in particular “loathe the cumbersome, time-consuming data entry that comes with using EHRs.” (\cite{CollierBurnout}). The frustration of using a new technology raises the cost of working in specific hospitals, which may lead physicians on the margin to make changes in labor market choices such as exiting the labor market altogether or shifting towards other work settings. However, the extent to which this frustration actually imposes a meaningful cost is unknown, making physician response an empirical question. Using a difference-in-differences research design, I estimate the effects of hospital EHR implementation on physician labor market outcomes.

My primary datasets are the CMS Shared Patient data and the Medicare Data on Provider Practice and Specialty (MD-PPAS), from which I construct a physician-level panel from 2009-2017 consisting of hospitalists and other general practice physicians who work exclusively in hospitals. The data captures whether a physician is exposed to EHRs in various hospitals, several labor market outcomes, and demographic information such as zip code, age, and gender. For all outcomes, the main independent variable of interest is a binary treatment variable capturing a physician's EHR exposure in hospitals. In the main analysis, a physician is considered to be exposed to an EHR if they are working in any hospital that implements an EHR. I estimate group time treatment effects of EHR exposure on the following physician decisions: (1) retirement, measured based on zero or missing billable activity in all future years of the panel, (2) where to physically work, measured by fraction of patients seen in an office setting, and list of primary zip codes, (3) number of patients seen, and (4) billing activity. 

Contrary to previous research stating that physicians do not retire based on work environment (\cite{Bahrami2002}), I find that EHR implementation led to a .005 ppt increase (25\% increase relative to the mean percentage) in the likelihood of a senior physician retiring in the year after implementation. Younger physicians (less than 60) are .002 ppts more likely to retire in the second year after exposure, possibly pointing to career switching. Whether physicians are retirement age or not, these results suggest that EHR implementation may have intensified access to care issues for patients, and that there was a surge of experienced physicians who left the labor force and were no longer available to influence physicians learning under them. In those who did not retire, I find that younger physicians are .015 ppts more likely to work in an office in the year of exposure to an EHR, but that exposure does not change the average fraction of patients seen in office settings for either age group. I observe similar trends in whether physicians change zip codes, although pre-trends may be driving this result. Finally, I find that both the total number of patients and billing activity increases in younger physicians in the first 2-3 years after exposure to an EHR. The ratio of billing activity to patient count increases less than the sample average, suggesting that billing activity did not dis-proportionally increase. 

A key assumption that underlies the main analysis is that physicians are constrained to using the technology in a hospital who chooses to adopt. I consider two possibilities where this assumption is violated. First, I investigate the existence of hospital employees with the purpose of using or assisting with an EHR on behalf of a physician. If this is occurring, the results are at least partially driven by the hiring of new employees. I find that the majority of hiring of data assistants took place in 2013-2014, a two year delay from the majority of EHR implementation. Further, my main results are not sensitive to the exclusion of these years and estimating the effect of EHR implementation only in hospitals without data assistants yields noisy, but similar, results. Second, I investigate whether the decision for hospitals to implement EHRs is exogenous to individual physicians. That is, a physician does not choose their own EHR exposure. While this assumption is reasonable in many settings, there may be instances where a physician is involved in hospital decision making. For this reason, I also consider a sensitivity analysis in which I limit the sample only to hospitals where vertical integration between physician and the hospital is low, as this indicates a lower probability that EHR implementation is endogenous. The results of the additional analyses indicate that my main findings are not sensitive to endogeneity concerns stemming from joint decision making among physicians and hospitals regarding EHR adoptions.


My analysis contributes to the literature examining effects of health information technology on health care. There is a robust group of empirical papers that examine the effects of EHRs on patient outcomes and hospital costs prior to the HITECH Act taking effect. Despite a large number of case studies that find generally positive effects (improved patient outcomes and decreased costs) (surveyed in \cite{Buntin2011TheResults}), more recent empirical work has found a mixture of results: that the median patient experiences no change due to EHRs (\cite{Agha2014TheCare}, \cite{McCullough2016HealthCoordination}, \cite{Meyerhoefer}) while newborns and severe patients experience improvement in health outcomes (\cite{Miller2009}, \cite{Freedman2015}, \cite{McCullough2016HealthCoordination}), and that hospital costs only decrease 6 years after implementation, if at all (\cite{Agha2014TheCare}, \cite{dranove2014trillion}). More relevant to EHR effects on physicians, a number of studies consider the effect of EHRs on productivity in specific settings. One study finds that nursing home productivity increases after adopting health IT (\cite{Hitt2016}), but another finds that physician productivity decreases by 11 percent due to EMRs being adopted in primary-care sites (\cite{Meyerhoefer}). 

I expand and improve upon this literature, first by considering a different mechanism by which EHRs could indirectly affect patients through various physician responses. This is the first study to my knowledge that connects EHR implementation to physician labor market decisions, and focuses on the difference in response across age groups. Further, I improve on the literature by considering the primary time period in which EHRs were rapidly implemented. This provides sufficient variation in the timing of EHR implementation and ensures the EHRs are advanced enough to be considered "meaningful" by the government's standards, and are likely affecting physicians' daily life. Similarly to past studies, I consider EHR implementation as a treatment variable in a difference-in-difference framework, but I improve on this by estimating average group time treatment effects for multiple years after implementation, which avoids common problems that arise in typical two way fixed effects estimation of heterogeneous effects.  




\section{Background on Electronic Health Records}

EHRs have been an important feature of health care since the 1980s. Early in the technology's existence, health care professionals perceived the technology as a complement to paper records, primarily deployed by large academic medical centers to improve efficiency in billing and/or scheduling. Physicians did not interact directly with these early-generation EHRs, and thus were not drastically affected by their implementation. As technological innovation made computers more portable, the usability of EHRs increased, creating what is known as the "physician workspace": a computer station for a physician to interface directly with an EHR to record patient updates. Despite usability, physicians kept the view that EHRs were purely complementary to paper due to burdensome data entry. Automation in data entry was non-existent, making it extremely time consuming for the user. Even when automation was developed for particular machines that performed monitoring, the hospital still held responsibility for the accuracy of information and therefore required physicians to manually check each data entry (\cite{evans2016electronic}). 

The HITECH Act was passed in 2009, designating \$27 billion in government subsidies to entities who used EHRs according to certain guidelines. The guidelines include having at least 80\% of patients in the system, regularly recording answers to specific questions, and protecting the security of the system to ensure privacy. The U.S. allocated subsidies according to stages of "meaningful" use: stage 1 (2011) focused on data collection, stage 2 (2012) extended to using the EHR for care support, and stage 3 (2014) extended to data sharing between practices. This program was successful in  spurring EHR adoption, shown in a 75 percentage point increase in the number of hospitals with EHRs from 2009 to 2015 (\cite{stats}). Figure \ref{fig:meanuse} shows a geographical comparison of U.S. hospitals who have received stage 1 of the meaningful use subsidy in 2011 vs. 2013, revelaing the nationwide and rapid expansion of technology. 

Figure \ref{fig:EPIC} shows a two computer screenshots of what physicians might see in their EHR system. On a daily basis, a physician spends approximately 23.7\%, 17\%, and 15.5\% of their time on documentation, chart review, and inbox management, respectively (\cite{arndt2017tethered}). These are the most time consuming portions of EHR use, according to the same study. In all, if a physician works for 12 hours, they spend roughly 3.2 hours with patients and 5.9 hours interfacing with an EHR. Despite a large amount of time spent with them, physicians still continue to report frustration over EHR use. The most time consuming functions are also reported as the most frustrating (\cite{dymek2021building}). A physician still likely spends almost 2 hours a day managing their inbox: deleting duplicate messages, sifting through messages meant for other members of a care team, and searching relevant information (\cite{dymek2021building}). Another common issue physicians face is the lack of usability of EHR systems, that functions either take minutes to locate or even longer to load. Hospitals often provide training upon adoption, but rapid system upgrades and changes make familiarity with the system difficult. Physicians have been voicing concerns over EHR usability since their beginning, indicating that they are likely responding in their behavior as well. 

\section{Data}

Using various linked data sets, I construct a physician-level panel spanning from 2009-2017 which measures physicians' exposure to EHRs over time, outcomes related to physician practice location and billable activity, and other relevant physician, hospital, and practice characteristics. I describe the different data sets used to construct the panel below. I include a detailed outline of data and variable construction in Appendix \ref{app:data}.

\subsection{MD-PPAS}

The Medicare Data on Provider Practice and Specialty (MD-PPAS) is a database of physicians which details claim counts, physician specialty, and practice location from 2009-2017. Included in the MD-PPAS is physician NPI, primary and secondary specialties, both patient and total billing counts in up to 12 different zip codes, and fraction of patients in different settings such as office, inpatient or emergency room. I first limit the sample of physicians based on specialty type. My analysis relies on EHR implementation in hospitals. Thus, I only include physicians who reported their primary specialty as hospitalist (physician who self-identifies as hospitalist or has 90\% of patients in inpatient setting), internal medicine, pediatrics, general practice or family practice in at least one year. However, I exclude physicians who list themselves as a specialist in all but one year. Since the focus of the analysis is physicians working in hospitals, I also drop physicians with less than 20\% of patients in a hospital setting in all years.

This data includes information on physicians which I use to construct the dependent variables for my analysis. Generally, these include the decision to retire (leave clinical work), the decision to switch to a different work setting, number of patients seen, and overall billing activity. I describe these outcome variables and their construction in detail below in Section \ref{sec:outcome}.


\subsection{Shared Patients}

CMS Shared Patient data records annual information on the number of patients who bill two entities under Medicare from 2009 to 2015. For example, if a primary care physician refers a patient to a specialist, then those two physicians have that particular shared patient in common. The number of shared patients are collected in 30, 90, or 180 day intervals. I focus on the number of shared patient billing two entities in the same day, within a 30 day interval. I employ this data to detect physician who work closely with hospitals and which of those hospitals have implemented an EHR, denoting each physician's exposure to EHRs. This measure of exposure defines various treatment variables for my analysis. I limit the entities by tax-code to only include shared patients for physician-hospital pairs, where the physicians in these pairs match the physicians from MD-PPAS. I am specifically interested in primary care physicians who have a close working relationship with hospitals, who do rounds within at least one hospital. The types of physicians are already limited to primary care, but there still may be office-based primary care physicians in the shared patient data. Therefore, I limit the sample of pairs based on a threshold of same day patients which depends on the number of years the physician appears in the data. I only include physician-hospital pairs who have non-missing data with at least 30 shared patients per year, which gives a threshold of work relationship without excluding physicians who have no claims for a number of years.

\subsection{AHA Survey Data}

Using hospital NPI, I link the physician-hospital pairs from the shared patient data to the American Hospital Association (AHA) survey, which contains information on hospital-level EHR use and other characteristics. I record the first year a physician is exposed to an EHR based on implementation in the hospitals they share patients with. I then aggregate this data to the physician level. Since the Shared Patient data does not extend to 2017 as the MD-PPAS data does, in the analysis I drop physicians who are not treated by 2015 as I do not observe whether they become treated in 2016 or 2017. That is, the data contains no units which are "never-treated".

\subsection{Physician Compare}

Finally, I include a physician's graduation year from Physician Compare. I limit to physicians who graduated before 2005, as anyone who graduated medical school after that will be finishing residency during the span of the data and will exhibit labor market changing behavior which could be correlated to EHR exposure purely because of switching work settings during a time of rapid EHR implementation.

\subsection{Outcome Variables}\label{sec:outcome}

The first dependent variable I consider is the decision for a physician to retire, or more broadly to stop practicing in a clinical setting altogether. Retirement is constructed as follows: 
$$\text{retire}_{it}=\mathbbm{1}\{FC_{it}=0 \text{ and retire}_{i,t-1}=0\}, $$
where $FC_{it}=\sum\limits_{j=t+1}^T\text{(claim count)}_{ij}$ is all future claims for physician $i$ in year $t$. Retire is therefore an indicator variable set to one in the first year that a physician has no claim counts at any point in the future \footnote{Alternatively, I could also define retirement using future number of patients. I use claim count for a more conservative measure of retirement, but using patient count yields identical results. In Appendix \textcolor{red}{()}, I limit the sample years to 2009-2015 to ensure sufficient future year of zero claims.}.

Next, I consider whether physicians exhibit behavior that suggests they have shifted towards other work settings. I measure this in the data in three ways. First, I consider the fraction of patients a physician sees in an office-based setting, which is directly available in the data. Second, I construct an indicator variable equal to one if a physician sees a positive fraction of patients in an office-based setting. This binary measure of practice setting offers less variation but captures a more discrete changes in which physicians enter or leave the office-based setting entirely. Third, based on the physician's practice zip code, I form an indicator variable equal to one if the physician's list of zip codes changes from one year to the next. While no single variable directly measures physician movement across hospitals, these three outcomes collectively provide insight into whether physicians desire to shift away from EHRs.

Finally, I consider outcome variables which explore inputs to overall efficiency in health care, the number of patients seen and total billing activity.


\subsection{Summary Statistics}

I show summary statistics for the entire sample in Table \ref{tab:sumstats1}. Only 2\% of physicians (approximately 750) retire over the course of the panel. Just over half of the physician-year observations have value one for working in an office in some capacity, with about a third of the number of patients being seen in an office setting. By construction, all physicians are exposed to an EHR at some point in the sample, so the variation in treatment variables comes from the timing of exposure. Physicians are typically exposed about one fourth of the way through the sample, but there is a positive number of physicians first exposed in each year from 2009-2015. The average physician is 48 years old, works with 1.5 hospitals and 1.3 systems. 


I also include a table of means comparing physicians younger than 60, at least 60, and physicians who ever retire (who can fall in either age category) in Table \ref{tab:splitstats}. Senior physicians see more patients and work in more hospitals on average than younger physicians. These older physicians also see a larger fraction of patients in an office setting and are more likely to work in an office in general. Finally, physicians in the higher age bracket are less likely to switch zip codes. The age groups are exposed to EHRs in similar ways.


In Figure \ref{fig:treatmentgraph}, I present a graph of the variation in treatment over time. In 2009, 32\% of the sample of physicians were already exposed to an EHR. That is, three quarters of the sample had no affiliation with EHRs at the beginning of the sample period. Since I drop any physicians who do not have exposure to an EHR by 2015, 100\% of the sample is exposed by then. The black line represents the average fraction of a physician's hospitals which are utilizing EHRs over time. This shows that, along with becoming first exposed over the sample period, physicians are also seeing EHRs implemented in more of their own hospitals over time.


\section{Empirical Strategy}

In this setting, treatment effects are likely heterogeneous across physicians who are exposed in different years since the underlying characteristics of hospitals who adopt in different years may be correlated with the decision to adopt. Therefore, to avoid negative weighting issues this causes in a two way fixed effects specification, I estimate average treatment effects for a specific group $g$ at time $t$: 
$$ATT(g,t)=\mathbbm{E}[Y_t(g)-Y_t(0)|G_g=1],$$
where $G_g=1$ for those in group $g$. A group indicates all physicians treated, or first exposed to an EHR, in the same year. To estimate the heterogeneous treatment effects, I employ the estimator established in \citeauthor{callaway2021difference} (\citeyear{callaway2021difference}). Other estimators that similarly address the concerns yield similar results, presented in Appendix ().

There are several assumptions necessary to identify $ATT(g,t)$. First, I assume that treatment is not reversed once it occurs. That is, once a physician is exposed to an EHR, they cannot be un-exposed. At the hospital level, this assumption is supported by both the institutional details of hospital EHR implementation and the definition of exposure. An EHR is a costly technology and requires a significant amount of collaboration to implement. A hospital does not have incentive to un-implement an EHR once it is implemented. They may add features or switch vendors, but do not reverse EHR use. This institutional detail is supported by Figure \ref{fig:hosp_treat}, which shows EHR status of a random sample of hospitals over time. Further, since we think of treatment at the physician level as exposure to the technology, this assumption is satisfied when thought of as whether the physician has ever used an EHR. Once they are exposed to the EHR for the first time, that experience cannot be forgotten. However, when considering the outcomes of patient count and billing activity, it could be that a physician moves from a hospital that uses an EHR to a hospital that does not, affecting the number of claims observed in the data. I investigate this possibility and limit the data to address it in Section \ref{sec:patientcount}. 

Second, I assume physicians do not anticipate EHR exposure prior to occurrence. Since the technology has different vendors and varies in capabilities, anticipation may be a concern if physicians learn that the system will be implemented and then do not actually use it until a future period. Generally, even a complex EHR system can be completely set up within a year, and most systems take 6 to 9 months (\cite{uzialko_2021}). Therefore, I proceed in the main specification assuming no anticipation. However, I explore and present results for a one year anticipation period in Appendix (). 

As is usual in a difference-in-differences framework, I assume a version of parallel trends based on not-yet-treated units. I assume that, conditional on covariates, average outcomes for those treated in group $g$ would have followed a parallel trend as those in groups treated in later periods. This assumption would be violated under external conditions in certain years that may be correlated with labor markets, such as a major recession. The time period I study, 2009-2017, is reassuringly stable. In my analysis, I present p-values for a Wald test of pre-trends. In most cases, there is no evidence of a pre-trend, however, I investigate the assumption in detail in Appendix (). 

An institutional assumption that I make is that physicians working in EHR-adopted hospitals are constrained to use the technology. My outcomes have two key determinants: the choice of physicians to learn and utilize the technology, and the realized effect of electronic health records themselves. The point of this paper is to identify the second determinant, and thus I assume that when a physician continues to work in an electronic record utilizing-hospital, they utilize the electronic health record system fully. That is, there are no physicians who stay actively working in a hospital but choose not to partake in the electronic health record used by the hospital. It is reasonable to believe that physicians are fully utilizing the EHRs in the hospital they work in, since by 2015 electronic health records were widely implemented and considered unavoidable. An objection to this assumption may be that physicians stay in a hospital, but the hospital hires an employee to do technology work for the physician. In this case, the effects seen could be from having the data assistant instead of EHR use. I investigate this in Section \ref{sec:dataass}.







\section{Effect of EHR Exposure on Labor Market Outcomes}

\subsection{Retirement}

Workers in developed countries tend to plan for formal retirement well in advance. An exogenous shock to a worker is not expected to change retirement age because of the amount of wealth and planning necessary to formally leave the labor force. However, physicians typically make this decision differently than workers in other industries. Most physicians plan to retire at age 60, but do not actually retire until age 69 (\cite{collier2017challenges}). When the time to retire comes, many physicians hesitate to abandon patients they have seen over the course of their career. The decision to delay retirement is not financial, but altruistic. Thus, retiring may be in a physician's choice set years before the realized decision to leave the labor force. Formal retirement (leaving the labor force altogether) is not the only way for physicians to transition out of a clinical role. There are opportunities for physicians of any age to switch careers towards more administrative roles such as teaching, consulting, or hospital management. While it is unlikely that physicians under 55 are formally retiring, physicians of any age could be career switching. The term "retirement" for the purpose of this paper indicates no longer seeing patients, and makes no assertion about what the physician does afterwards. A physician is said to retire in a given year only if it is the first year in which the physician has no future Medicare claims.

As an exogenous shock to work environment, EHR implementation may lead to an earlier retirement than would have otherwise occurred. Media articles focusing on interviews with particular physicians suggest that the implementation of EHRs did in fact lead to retirement in older physicians (\cite{ringel_2019}, \cite{loria_2020}). The aim of this section is to test that hypothesis and evaluate whether younger physicians exhibited similar behavior.

In Figure \ref{fig:retirefirst}, I present aggregated group time treatment effects of being exposed to an EHR in any hospital on the likelihood of retirement for the full sample of physicians, as well as split samples for those at least 60 and less than 60 years of age. The top panel suggests that for any age physician, being exposed to an EHR leads to a .002 ppt increase in the likelihood of retiring in both the first and second year after after implementation. While this result is numerically small, it is statistically and economically meaningful relative to the proportion of physicians who actually retire in the sample, .02. Thus, relative to the mean, this effect represents a 10\% increase in the likelihood of retirement due to EHR exposure. Considering the magnitude of the decision to retire, it would be alarming to see a numerically large effect on this outcome. A 10\% increase in the likelihood of such a decision is a meaningful jump. While the presented estimates are noisy, the rarity of retirement in the sample combined with the observed precision in the estimates indicates that a positive effect on the likelihood of retirement did occur as a result of EHR implementation.


Next, I examine how the results differ when considering physicians in different age brackets: pre-retirement and typical retirement age. These results are shown in the bottom panels of Figure \ref{fig:retirefirst}, and suggest that senior physicians are driving the positive estimate found above. A physician who is at least 60 years old is .005 ppts more likely to retire after being exposed to an EHR, a 25\% increase relative to the mean. In comparison, physicians less than 60 are far less likely, if at all, to respond to EHR exposure in the first year after EHR exposure. Physicians in the young category have a larger effect on the likelihood of retiring in the second year after exposure relative to the first year. While I do not have data to test this, if young physicians are more likely to switch careers after EHR exposure then it may take one additional year to prepare for that decision relative to retirement.   



\subsection{Work Setting}

For most physicians, EHRs will not impose a costly enough burden to induce retirement. However, there are other ways physicians might change their labor market behavior under sufficient incentive to avoid the technology. I consider the decision to switch work settings using multiple outcomes. First, I directly investigate the extent to which physicians work in office based setting and whether this changes due to EHR exposure. I use variation in two different variables: (1) an indicator variable equal to 1 if the physician has seen any positive number of patients in an office in a given year, and (2) the fraction of patients a physician sees in an office based setting in a given year. This analysis does not include physicians who retire at any point in the sample, as a zero could indicate dropping out of the data instead of working solely in a hospital. 

I first discuss the effect of EHR exposure on the probability of working in an office setting, shown in Figure \ref{fig:officefirst}. The top panel shows aggregated group time effects for physicians of any age, and suggests that EHR exposure increases the likelihood of working in an office in the year after exposure by .015 percentage points, a 2.6\% increase relative to the mean. This is positive, but somewhat inconsequential effect, both economically and compared to the effect of EHRs on retirement discussed above. Splitting the sample by age reveals that this result is driven by the younger sample of physicians. There is no evidence that older physicians are shifting towards offices. This may be due to senior physicians already being established in some capacity in offices prior to EHR exposure, but is also supported by intuition that older workers are more "stuck" to their particular work environment. 

If senior physicians are already established in offices in some capacity, they may be more affected on an intensive margin, the fraction of patients seen in an office relative to hospital or other settings. I show the estimates for this outcome in Figure \ref{fig:officesecond}, where I find no evidence that EHR exposure affects the fraction of patients seen in office settings. This result has varying implications depending on how total number of patients is affected. If total number of patients remains constant, then the absolute number of patients seen in offices is also unchanged. However, if the total number of patients seen (which depends on efficiency in hospitals also) changes, then the absolute number of patients seen in offices necessarily changes in the same direction. I revisit the implications of this result in Section \ref{sec:patientcount}, where I examine the total number of patients as an outcome variable.


The previous results only capture switching behavior which involves office-based settings. A more general way to examine physician switching behavior is to investigate whether the zip codes in which the physicians work is changing, which captures both office and other switching behaviors. Again, I exclude physicians who ever retire. The outcome variable is equal to 1 if the list of primary zip codes (up to twelve of them) changes. This variable is limited in the way it captures switching behavior because multiple scenarios are flagged as changing zip codes: leaving a particular zip code, adding one new zip code but leaving the rest unchanged, or completely changing zip codes. While this is a vague description of behavior, it provides a different insight into changes occurring because of EHR implementation.

The effects of EHR exposure on zip code switching over time are shown in Figure \ref{fig:zip}. The graph behaves similarly to the probability of working in an office in that the probability of switching zip codes increases and the result is driven solely by the younger demographic. However, this result is meaningfully larger; a physician is .025 ppts more likely to switch zip codes in the same year as EHR exposure and the year after, a 31.25\% increase relative to the mean. This result suggests that EHRs are imposing a cost high enough for physicians to exhibit significant switching behaviors, whether that is towards offices or other hospitals.




\subsection{Patient Count and Billable Activity}\label{sec:patientcount}

An important, and heavily debated, aspect of the implementation of EHRs is whether or not they allow users to be more productive. A main purpose of the HITECH Act was to improve efficiency of care, specifically to decrease time spent on administrative burden. Such a decrease implies a change in patient care: either a larger number of patients have access to an appointment, or the same amount of patients receive more face-to-face time with physicians during appointments. Alternatively, because the technology is not hugely user friendly, it may not decrease administrative burden after all. In this section, I address EHRs' effect on productivity by estimating its effect on a physician's number of patients seen. 

The effect of EHR exposure on patient count is shown in Figure \ref{fig:patient}. For physicians of any age, patient count increases by 11.5 patients in the year of EHR exposure, 24 patients in the year after exposure, 15.3 patients in the second year after exposure, and then no effect after the second year. These results translate to a .5-1\% increase in the number of patients seen relative to the mean. As with other outcomes, this result is totally due to changes by the younger group of physicians. While there is a statistically larger effect for younger physicians relative to senior physicians (who likely take longer to adapt to technology changes), the observed effect is not largely meaningful. 

This variable is also driven by demand for health care. A major change in the U.S. health care system, the Affordable Care Act (ACA) was passed in 2010 and took effect in 2014 for the majority of states that partook. If my results are driven primarily by the group of hospitals who implemented EHRs in 2014, the ACA may be driving the results. The ACA could potentially affect patient counts in opposite ways: if Medicare patients are crowded out by an increased demand in Medicaid patients, or if the small changes to Medicare led to an increase in utilization from the Medicare population. A recent study found no negative spillovers to the Medicare population due to the ACA (\cite{carey2020impact}). Further, I present average group time treatment effects for each treated group in Figure \ref{fig:patientgroup}. Each group shows the same pattern of a small increase in patient count in the first 0-2 years after implementation. 

Total patient count is directly related to the fraction of patients seen in an office setting. For younger physicians, since the total number of patients increases in the first couple of years after exposure, the absolute number of patients seen in offices increases as well. That is, it seems that younger physicians are shifting a positive amount of work towards offices but still able to increase productivity in the time spent in the hospital, thus the null effect on fraction of office patients. 

Further, EHRs also have the potential to change billing activity differentially than patient count changes. If, for example, EHRs reduce the administrative burden of filing extra claims, then physicians might increase the things they bill for whether the amount of patients has changed or not. On average for the entire sample, claim counts are roughly 5.5 times the number of patients seen in a year. Shown in Figure \ref{fig:claim}, young physicians increase claim count by 37 in the year of exposure and 96 in the year after exposure, 3-4 times the increase in patient count. Thus, billing activity increases proportionally less than average. EHRs do not seem to be facilitating more billable activity. 

However, for both patient and claim count, the results include physicians who shift place of work, which may lead to biased estimates since a physician could switch hospitals due to an EHR and then change practice behavior due to the move, not due to the EHR. Thus, I limit the sample to only physicians who work stay with the same one hospital through the entire sample. These physicians do not retire or see patients in another hospital. Thus, when an EHR is implemented, the physician remains utilizing the EHR for the remainder of the period. I present the results for this limited sample in Figure \ref{fig:limitedsample}. 


\subsection{Late Adopters as Comparison Group}

In all outcomes, if EHR exposure affects physician behavior, the effect occurs in the first 0-2 years after exposure. All specifications point to the effect of EHR exposure to become null in the 3-4 years after exposure. This is likely due to the composition of the hospitals adopting at different times. Consider time $t+4$. The only hospitals to analyze the effect in this period are those who adopted in 2010 compared to hospitals who adopted in 2015. The group of hospitals who adopted in 2015 may be compositionally different than early adopters if there is a characteristic reason for late adoption such as demographic of patients. For this reason, I focus discussion of results on the immediate years after exposure when hospitals the comparison groups are more comprehensive. 


\section{Physician EHR Avoidance}

\subsection{Hiring Data Assistants}\label{sec:dataass}

One concern with the presented analysis is that any effect could be caused by an outside factor also correlated with EHR adoption in hospitals. One such occurrence would be the existence or hiring of employees with the purpose of utilizing electronic health records on a physician's behalf, which is a way for physicians to continue working in a hospital with an EHR but not actually use the EHR. Employees of this type are traditionally referred to as scribes, and are assigned a medical tax ID when working in hospitals. they have the following titles: Coding Specialist (Hospital Based), Health Information Technologist and Registered Record Administrator. I will refer to any employee in these categories as a data assistant. In this Section, I investigate whether data assistants were hired in accordance with EHR implementation and whether the presence of data assistants affects physician response to EHRs.  

First, using NPPES data on all NPIs and their tax information, I analyze the existence of data assistants in hospitals. Figure \ref{fig:dataassistant_histogram} presents a frequency plot of the year of activation for every tax code that falls in the categories listed above. The total number of these registered employees is 875, and the first data assistants ever registered were in 2005. From 2005 to 2013, 15-60 more employees were registered each year. The graph is clearly skewed towards later years, where a significant increase in the number of new data assistants occurred in 2014. If hiring data assistants was directly correlated with both EHR implementation and physician frustration, one would expect the increase to occur from 2011-2012, when a majority of physicians first became exposed to EHRs. My results are not sensitive to leaving out later years, which is a good indication that the results are not driven by the enumeration of data assistants. 

Further, I investigate the occurrence of hospitals and data assistants sharing patients in the CMS Shared Patient data. I create a list of hospital-years for which the hospital has a positive number of shared patients with a data assistant. I merge this to the physician-hospital pairs and create a physician level variable for whether any hospital worked with is associated with a data assistant. I find that .1\% of physician-year observations are associated with data assistants. This sample of physicians is too small to analyze whether data assistants are more likely with EHR exposure, yielding largely noisy estimates. However, I limit the sample to only physicians not associated with data assistants and find that the magnitude of estimates are similar, albeit more noisy. Details of this analysis are given in Appendix (). 



\subsection{Physician Influence of EHR Decision in Hospitals}


\section{Conclusion}

In this study, I analyze the effect of implementing a now-common technology in hospitals, electronic health records, on various physician labor market outcomes. This technology has been controversial and debated as a burdensome load on physicians that may contribute to physician burnout, and therefore patient outcomes. 

Using an event study framework, I find that physicians of all ages are more likely to stop clinical work altogether when exposed to an EHR in a hospital, where physicians of retirement age have an even larger increase relative to younger physicians. Further, I find that younger physicians are more likely to work in an office based setting after EHR exposure, and physicians who remain in the labor force increase both number of patients seen and billing activity, but billing activity per patient decreases with EHR exposure. 

The implications of these results are threefold. First, a surge of experienced physicians left the labor force in a concentrated number of years due to EHR exposure, potentially worsening access to care. In a field where younger physicians learn a lot by watching those more experienced than them, this particular loss may have long term negative effects on patient outcomes, but this is a question left to future research. Further, physicians are exhibiting behavior that suggests they incur switching costs to avoid burdens externally placed on them. This has implications when considering the current policy movement towards regulated care, that physicians may go to extreme measures to avoid such regulation. Finally, these results speak to the ongoing debate of whether EHRs are generally beneficial to its users. While the technology may be a burden, physicians are able to see more patients because of their use. 

This paper contributes to our understanding of health information technology by providing empirical evidence of the theoretically ambiguous physician response to electronic health records being rapidly implemented in hospitals. This contribution is relevant to EHR vendors, hospitals, and policymakers who are involved in further implementation of complex technology. 



\section{Tables and Figures}

\begin{figure}[htp]
    \centering
    \caption{Hospitals Receiving Meaningful Use Stage 1 Subsidy}
    \includegraphics[scale=.6]{Objects/QS-Hospitals-Receiving-Payments-for-MU-and-Adoption.png}
    \caption*{Source: HealthIT.gov}
    \label{fig:meanuse}
\end{figure}

\begin{figure}[htp]
    \centering
    \caption{Screenshot of EHR System}
    \includegraphics[scale=.5]{Objects/epic-ehr-screenshot.jpg}
    \includegraphics[scale=.135]{Objects/EHRimage2.jpg}
    \label{fig:EPIC}
\end{figure}

\import{Table Code}{overall stats.tex}

\import{Table Code}{split stats.tex}

\begin{figure}[p]
\centering
    \caption{Treatment Variables Over Time}
    \includegraphics[scale=.55]{Objects/sum_stats_year.pdf}
    \label{fig:treatmentgraph}
\end{figure}

\begin{figure}
    \centering
    \caption{Hospital EHR Adoption Over Time}
    \includegraphics[scale=.8]{Objects/hosp_treat.pdf}
    \label{fig:hosp_treat}
\end{figure}

\begin{figure}[p]
    \centering
    \caption{Effect of EHR Exposure on Retirement}
    \includegraphics[scale=.4]{Objects/retire_plot.pdf}
    \label{fig:retirefirst}
\end{figure}

\begin{figure}[p]
    \centering
    \caption{Effect of EHR Exposure on Likelihood of Working in Office}
    \includegraphics[scale=.4]{Objects/officeind_plot.pdf}
    \label{fig:officefirst}
\end{figure}

\begin{figure}[p]
    \centering
    \caption{Effect of EHR Exposure on Fraction of Total Patients Seen in Office}
    \includegraphics[scale=.4]{Objects/officefrac_plot.pdf}
    \label{fig:officesecond}
\end{figure}

\begin{figure}[p]
    \centering
    \caption{Effect of EHR Exposure on Likelihood of Changing Zip Codes}
    \includegraphics[scale=.4]{Objects/zip_plot.pdf}
    \label{fig:zip}
\end{figure}

\begin{figure}[p]
    \centering
    \caption{Effect of EHR Exposure on Patient Count}
    \includegraphics[scale=.4]{Objects/patient_plot.pdf}
    \label{fig:patient}
\end{figure}


\begin{figure}[p]
    \centering
    \caption{Effect of EHR Exposure on Patient Count by Group}
    \includegraphics[scale=.3]{Objects/patient_group.pdf}
    \label{fig:patientgroup}
\end{figure}

\begin{figure}[p]
    \centering
    \caption{Effect of EHR Exposure on Claim Count}
    \includegraphics[scale=.4]{Objects/claim_plot.pdf}
    \label{fig:claim}
\end{figure}

\begin{figure}[p]
    \centering
    \caption{Effect of EHR Exposure on Productivity, Limited Sample}
    \includegraphics[scale=.4]{Objects/limitedsample_plot.pdf}
    \label{fig:limitedsample}
\end{figure}

\begin{figure}[p]
\centering
\caption{Frequency of Data Assistant Enumeration by Year}
\includegraphics[scale=.5]{Objects/dataassistant_histogram.pdf}
\label{fig:dataassistant_histogram}
\end{figure}

\clearpage

\renewcommand*{\bibfont}{\footnotesize}

\printbibliography

\clearpage


\appendix

\section{Data}\label{app:data}

This section details the process of creating the data set used in the analysis of the main paper. 

\subsection{Physician Specialties by Tax Code}\label{sec:taxcode}

I begin by connecting every National Provider Identifier (NPI)\footnote{Data can be downloaded from \hyperlink{https://download.cms.gov/nppes/NPI/Files.html}{\text{https://download.cms.gov/nppes/NPI_Files.html}}} to a description of their tax code\footnote{Data can be downloaded from \hyperlink{https://nucc.org/index.php/code-sets-mainmenu-41/provider-taxonomy-mainmenu-40/pdf-mainmenu-53}{\text{https://nucc.org/index.php/code-sets-mainmenu-41/provider-taxonomy-mainmenu-40/pdf-mainmenu-53}}}. I then categorize NPIs by key words in their tax code description. Any description containing ``Internal Medicine", ``Hospitalist", ``Family Medicine", ``General Practice", or ``Pediatrics" is classified as a primary care physician (PCP), and any description containing ``hospital" is classified as a hospital. I create two data sets, one containing only PCPs and one containing only hospitals. These data are saved to use in identifying the relevant NPIs in the CMS Shared Patient Data. 



\subsection{CMS Shared Patient Data}\label{sec:sharedpat}

For years 2009-2015, the CMS collected detailed information on the number of Medicare patients shared between any two NPIs within 30, 90, or 180 day intervals\footnote{Data can be dowloaded from \hyperlink{https://www.nber.org/research/data/physician-shared-patient-patterns-data}{https://www.nber.org/research/data/physician-shared-patient-patterns-data}}. I use data that captures shared patient activity in 30 day intervals. I merge the shared patient data to the filtered tax code data created in Section \ref{sec:taxcode} to identify NPIs who are either PCPs or hospitals. I filter the pairs to include one physician and one hospital; there are 12.6 million of these pairs. Some pairs are duplicates from the hospital being listed first or second in the shared patient data. I combine duplicates into one observation, summing the same day count variable. Once duplicated are removed, there are 7.1 million observations. Most of the pairs have very few shared patients, which is not indicative of the physician working inside the hospital. I drop any pairs who do not have at least 30 shared patients in the same day per year the pair appears in the data. The final list of pairs consists of 1.3 million observations. 

\subsection{Pair-Level Variables}

I combine the physician-hospital pairs created in Section \ref{sec:sharedpat} with various data sets for hospital or physician level information. First, I merge to CMS Physician Compare\footnote{Data can be downloaded from \hyperlink{https://data.cms.gov/provider-data/dataset/mj5m-pzi6}{https://data.cms.gov/provider-data/dataset/mj5m-pzi6}} for information on each physician's graduation year and gender. This data contains more information on physician quality, but I am limited to time-invariant information since Physician Compare spans 2012-2015 and 2010-2012 is a time of major EHR implementation. I drop any physicians who graduated medical school after 2004, since graduating after that means leaving residency or graduating during the span of my main data, 2009-2017, and potentially exhibiting labor market changes that seem associated with EHRs but are not. 

For hospital level variables, I use an AHA-NPI crosswalk to merge the pairs to the Annual Hospital Administration Survey (AHA Survey) from 2009-2015. The variables I include from this data are a hospital's number of beds, organization type, system ID, and EHR use. There are 4,253 unique AHA hospitals in the shared patient data. I drop any hospitals that aren't in the AHA Survey due to lack of information on EHR use. After limiting the hospitals, there are 780,000 observations of pair-years left. Next, I investigate the hospitals with missing information for EHR use. There are 89 hospitals in the data who never answer the survey question about EHR use; I drop these hospitals. If a hospital does not answer the survey question in one year, but the year before and after have an identical survey answer, I fill in the missing year of information. Then there are 4,253 observations with missing data for the EHR survey question. I make a further limitation to hospitals with at least 10 beds. There are only 58 unique hospitals in the data with less than 10 beds, and these are likely very different from the rest of the sample of hospitals. Then, I create a physician-level variable which captures the average number of beds in the hospitals they work with in each year. I sum a physician's same day count over their hospitals in a given year to create a physician level patient count variable. I also create this variable only for hospitals who are using an EHR and only for hospitals who are not using an EHR.

I merge the remaining pairs to an additional survey, the AHA IT Supplement, only sent to hospitals who reported using an EHR. There are more detailed questions in this survey including the EHR vendor, and the ranking of documentation and decision making features the hospital uses. There is a lot of missing data in this survey, which is why I do not use it in my main analysis. 

I use the general survey question about each hospital's EHR use to define my key treatment variable. The survey asks the extent to which the hospital utilizes an EHR and allows for three answers: not at all, partially utilize an EHR, and fully utilize an EHR. I create a binary variable equal to one if the answer is fully utilize, zero otherwise. The data will be aggregated up to the physician level due to physician level outcome variables, so I first transform the pair-level EHR variable into a treatment variable at the physician level. I create a variable capturing the first year that at least one of a physician's connected hospitals uses an EHR. For the endogeneity discussion in Section \ref{sec:endog}, I create a similar variable but only count a physician as exposed if a hospital categorized as "low-integration" adopts an EHR. The level of vertical integration between hospital and physician is defined as in Madison (\citeyear{madison2004hospital}) using the organization type of the hospital. Hospitals classified as IPA or PHO are low-integration, and any other type is high0integration. I also create a variable that captures the fraction of hospitals a physician works with that use an EHR, which varies on the dimension of EHR use and on number of hospitals worked with. Finally, the last variable I create before aggregating to the physician level is an indicator for whether the physician works with the same set of hospitals through the entire sample. I create this variable by comparing a physician's number of hospitals in each year to the maximum number of hospitals they ever work with. If there is any year in which the a physician's number of hospitals is less than the maximum number of hospitals they ever work with, the indicator for never having a new NPI is set to 0. I use this variable to limit the sample in Section \ref{sec:patientcount}. 

Finally, I aggregate to the physician level by only keeping variables that do not vary by hospital. The variables in the physician level data are year, physician NPI, graduation year, average number of beds in hospitals worked with, years of experience, number of hospitals, fraction of hospitals with EHR, minimum year exposed to EHR, minimum year exposed to EHR in low integration hospital, number of systems, patient count, patient count in EHR hospitals, patient count in non-EHR hospitals, and never works with a new NPI. I complete this data to include years 2016 and 2017, but leave time varying variables missing for those years. Thus, this is a balanced panel of physicians. I save this data to be merged with physician labor market activity in Section \ref{sec:appmdppas}.

\subsection{Physician Labor Market Activity}\label{sec:appmdppas}

Using physician NPI, I merge the physician treatment data to Medicare Data on Provider Practice and Specialty (MD-PPAS)\footnote{Information on data located at \hyperlink{https://resdac.org/cms-data/files/md-ppas}{https://resdac.org/cms-data/files/md-ppas}}, which psnas 2009-2017. This data contains variables on physician specialty, Medicare claim counts in various zip codes, unique number of patients seen, fraction of patients seen in specific settings, patient demographics, and physician date of birth. Once the data is merged, I make a further limitation to drop any physicians with less than 20\% of their total patients in a hospital setting. This limitation is to continue ensuring that I am focusing on physicians working in hospitals who will be subject to their EHR use. With this limitation, there are 403,000 observations left in the data. 

Next, I create the dependent variables used in the analysis. The first outcome is whether a physician chooses to retire or not. First, I sum a physicians claim counts across zip codes into one variable for total claim count in a given year. Then, I create a variable that sums up a physicians claims in all future years. In 2009, this variable sums up all claims from 2010-2017, and so on. Then I create a variable for the first year that a physician has a future claim count of zero. However, this counts retirement one year too early. For example, if a physician has no claims from 2014 to 2017, the minimum year variable is set to 2013, but the first year of retirement is 2014. Thus, I add one year to the minimum year variable. I then define the outcome variable, retire, as one if a physician ever retires and the year is equal to the first year of retirement. I utilize a physician level variable for whether they ever retire for sample limitations in analyzing other outcomes. 

The next outcome variables consider level of labor market activity in office settings. The first decision to make is how to handle years where a physician has missing data for claims. A year of missing data can mean the physician had no claims that year, or the observation was removed due to insufficient claims (less than 11). A conservative assumption is to assume a year of missing data means zero claims, thus, I set all missing claim counts to zero. The data already contains a variable for the fraction of patients seen in an office setting, which is used in the analysis, but I further create an indicator variable for whether the physician sees any positive amount of patients in an office setting. 

I also create an indicator variable for whether the physician switches zip codes from one year to the next. The physicians in this data work in up to 12 different zip codes, so I their zip codes into one string variable, a list of all zip codes in a year. I make this variable into wide format: zip list 2009, zip list 2010, ... , zip list 2017.  In year $t$, I define a variable change zip that is equal to one if the zip list in $t-1$ is different from the zip list in $t$, zero otherwise. Finally, the variables for unique patient count and claim count are already defined. This is the final data used in the paper, with 403,020 observations. 



\section{Sensitivity Analysis}

\subsection{Physician Selection into EHRs}\label{sec:endog}

The main specification of the paper defines a binary treatment variable for whether a physician is exposed to an EHR in any hospital they work in. A concern that one may have is that physicians choose their individual EHR exposure, making this an endogenous treatment variable. The results do not indicate that this is happening on average, since we would expect physicians not to respond on the margin of retirement or switching zip codes if they are choosing to be exposed to EHRs. Especially in larger hospitals, this is unlikely to occur as an individual physician will not have much input to high level decisions. However, as an attempt to provide more evidence of labor market changes as a result of EHR exposure, I provide results for a specification with an alternate binary treatment variable. This variable is whether a physician has been exposed to an EHR in a hospital considered to have low vertical integration with physicians. 



\subsection{Anticipation}


\clearpage

\section{Tables and Figures}





















\end{document}